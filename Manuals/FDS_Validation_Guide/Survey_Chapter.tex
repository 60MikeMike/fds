% !TEX root = FDS_Validation_Guide.tex

\chapter{Survey of Past Validation Work}

\label{Survey_Chapter}

In this chapter, a survey of FDS validation work is presented. Some of the work has been performed at NIST, some by its grantees and some by engineering firms using the model. Because each organization has its own reasons for validating the model, the referenced papers and reports do not follow any particular guidelines. Some of the works only provide a qualitative assessment of the model, concluding that the model agreement with a particular experiment is ``good'' or ``reasonable.'' Sometimes, the conclusion is that the model works well in certain cases, not as well in others. These studies are included in the survey because the references are useful to other model users who may have a similar application and are interested in even qualitative assessment. It is important to note that some of the papers point out flaws in early releases of FDS that have been corrected or improved in more recent releases. Some of the issues raised, however, are still subjects of active research. The research agenda for FDS is greatly influenced by  the feedback provided by users, often through publication of validation efforts.

It is useful to divide the various validation exercises described in this chapter into two classes -- those for which the heat release rate (HRR) of the fire is {\em specified} as an input to the model and those for which the HRR
is {\em predicted} by the model. The former is often the case for a design application, the latter for a forensic reconstruction.

Design applications typically involve an existing building or a building under design. A so-called ``design fire'' is specified either by a regulatory authority or by the engineers performing the analysis. Because the fire's heat release rate is specified, the role of the model is to predict the transport of heat and combustion products throughout the room or rooms of interest. Ventilation equipment is often included in the simulation, like fans, blowers, exhaust hoods, HVAC ducts, smoke management systems, etc. Sprinkler and heat and smoke detector activation are also of interest. The effect of the sprinkler spray on the fire is usually less of interest since the heat release rate of the fire is specified rather than predicted. Detailed descriptions of the contents of the building are usually not necessary because these items are assumed to not contribute to the fire, and even if they are, the burning rate will be specified, not predicted. Sometimes, it is necessary to predict the heat flux from the fire to a nearby ``target,'' and even though the target may heat up to some specified ignition temperature, the subsequent spread of the fire usually goes beyond the scope of the analysis because of the uncertainty inherent in object to object fire spread.

Forensic reconstructions require the model to simulate an actual fire based on information that is collected after the event, such as eye witness accounts, unburned materials, burn signatures, etc. The purpose of the simulation is to connect a sequence of discrete observations with a continuous description of the fire dynamics. Usually, reconstructions involve more gas/solid phase interaction because virtually all objects in a given room are potentially ignitable, especially when flashover occurs. Thus, there is much more emphasis on such phenomena as heat transfer to surfaces, pyrolysis, flame spread, and suppression. In general, forensic reconstructions are more challenging simulations to perform because they require more detailed information about the room contents, and there is much greater uncertainty in the total heat release rate as the fire spreads from object to object.

Validation studies of FDS to date have focused more on design applications  than reconstructions. The  reason is  that design applications usually involve specified fires and demand a minimum of thermo-physical properties of real materials. Transport of smoke and heat is the primary focus, and measurements can be limited to well-placed thermocouples, a few heat flux gauges, gas samplers, etc. Phenomena of importance in forensic reconstructions, like second item ignition, flame spread, vitiation effects and extinction, are more difficult to model and  more difficult  to  study with well-controlled experiments. Uncertainties in material properties and measurements, as well as
simplifying assumptions in the model, often force the comparison between model and measurement to be qualitative at best. Nevertheless, current validation efforts are moving in the direction of these more difficult issues.



\section{Validation Work with Pre-Release Versions of FDS}

FDS was officially released in 2000. However, for two decades various CFD codes using the basic FDS hydrodynamic framework were developed at NIST for different applications and for research. In the mid 1990s, many of these different codes were consolidated into what eventually became FDS. Before FDS, the various models were referred to as LES, NIST-LES, LES3D, IFS (Industrial Fire Simulator), and ALOFT (A Large Outdoor Fire Plume Trajectory).

The NIST LES model describes the transport of smoke and hot gases during a fire in an enclosure using the Boussinesq approximation, where it is assumed that the density and temperature variations in the flow are relatively small~\cite{Rehm:1,Rehm:SIAM83,Rehm:ANM85,Rehm:IAFSS3}. Such an approximation can be applied to a fire plume away from the fire itself. Much of the early work with this form of the model was devoted to the formulation of the low Mach number form of the Navier-Stokes equations and the development of the basic numerical algorithm.  Early validation efforts compared the model with salt water experiments~\cite{Baum:1,McGrattan:1,Rehm:IAFSS5},  and fire plumes~\cite{Baum:IAFSS5,Baum:2,Baum:3,Baum:4}. Clement validated the hydrodynamic model in FDS by measuring salt water flows using Laser Induced dye  Fluorescence~(LIF)~\cite{Clement:1}. An  interesting finding of this work was that the transition from a laminar to a turbulent plume is very difficult to predict with any technique other than DNS.

Eventually, the Boussinesq approximation was dropped and simulations began to  include more fire-specific  phenomena. Simulations of enclosure  fires  were  compared  to experiments  performed  by Steckler~\cite{McGrattan:4}. Mell et al.~\cite{Mell:1} studied small helium plumes, with particular attention to the relative roles of baroclinic torque and buoyancy as sources of vorticity. Cleary et al.~\cite{LES:6} used  the LES model to simulate the environment seen by multi-sensor fire detectors and performed some simple validation work to check the model before using it. Large fire experiments were performed by NIST at the FRI test facility in Japan, and at US Naval aircraft hangars in Hawaii and Iceland~\cite{Davis:1}. Room  airflow  applications  were  considered by  Emmerich  and McGrattan~\cite{Emmerich:1,Emmerich:2}.

These early validation efforts were encouraging, but highlighted the need to improve the hydrodynamic model by introducing the Smagorinsky form of large eddy simulation. This addition improved the stability of the model because of the relatively simple relation between the local strain rate and the turbulent viscosity. There is both  a  physical and  numerical  benefit  to the  Smagorinsky model. Physically, the viscous term used in the model has the right functional form to describe sub-grid mixing processes. Numerically, local oscillations in the computed flow quantities are damped if they become large  enough to threaten the stability  of the entire calculation.






\section{Validation of FDS since 2000}

There is an on-going effort at NIST and elsewhere to evaluate FDS as new capabilities are added. To date, most of this work has focused on the model's ability to predict the transport of heat and exhaust products from a fire through an enclosure. In these studies, the heat release rate is usually prescribed, along with the production rates of various products of combustion. More recently, validation efforts have moved beyond just transport issues to consider fire growth, flame spread, suppression, sprinkler/detector activation, and other fire-specific phenomena.

The validation work discussed below can be organized into several categories: Comparisons with full-scale tests conducted especially for the chosen evaluation,  comparisons  with previously  published full-scale test data, comparisons with standard tests, comparisons with documented fire experience, and comparisons with engineering correlations. There is no single method by which the predictions and measurements are compared.  Formal, rigorous validation exercises are time-consuming and expensive. Most validation exercises are done simply to assess if the model can be used for a very specific purpose. While not comprehensive on their own, these studies collectively constitute a valuable assessment of the model.


\subsection{Fire Plumes}

There are several examples of fire flows that have been extensively studied, so much so that a set of engineering correlations combining the results of  many experiments  have been  developed. These correlations are useful to modelers because of their simplicity. The most studied phenomena include fire plumes, ceiling jets, and flame heights.

Although much of the early validation work before FDS was released involved fire plumes, it remains an active area of interest. One study by Chow and Yin~\cite{Chow:1} surveys the performance of various models in predicting plume temperatures and entrainment for a 470~kW fire with a diameter of 1~m and an unbounded ceiling. They compare the FDS results with various correlations and a RANS (Reynolds-Averaged Navier-Stokes) model.

Battaglia et al.~\cite{Battaglia:1} used FDS to simulate fire whirls.  First, the model was shown to reproduce the McCaffrey correlation of a fire plume,  then it was shown to reproduce qualitatively certain features of fire whirls. At the time, FDS used Lagrangian elements to introduce heat from the fire (no longer used), and this combustion model could not replicate the extreme stretching of the core of the flame zone.

Quintiere and Ma~\cite{Ma:2,Ma:3} compared predicted flame heights and plume centerline temperatures to empirical correlations.  For plume temperature,  the Heskestad  correlation~\cite{SFPE:Heskestad} was chosen. Favorable agreement was found in the plume region, but the results near the flame region were found to be grid-dependent, especially for low $Q^*$ fires. At this same time, researchers at NIST were reaching similar conclusions, and it was noticed by both teams that a critical parameter for the model is $D^*/\dx$, where $D^*$ is the characteristic fire diameter and $\dx$ is the grid cell size. If this parameter is sufficiently large, the fire can be considered well-resolved and agreement with various flame height correlations was found. If the parameter is not large enough, the fire is not well-resolved and adjustments must be made to the combustion routine to account for it.

Guti\'{e}rrez-Montes et al.~\cite{Gutierrez:Building_and_Environment} simulated 1.3~MW and 2.3~MW fires in a 20~m cubic atrium using FDS version 4. Similar experiments were conducted at VTT, Finland, in a 19~m tall test hall with similar sized fires. These results are included in Section~\ref{VTT_Plume_Temperature}.

Hurley and Munguia~\cite{Hurley:GCR09-921,Hurley:JFPE2009} compared FDS (version 4) simulations with plume and ceiling jet measurements from a series of full-scale tests conducted by Underwriters Laboratories. The tests were conducted in a 36.6~m by 36.6~m compartment with ceiling heights ranging from 3~m to 12.2~m. Heat release rates followed a modified t-squared growth profile. Thermocouples attached to brass disks were used to simulate thermal detectors.


\subsection{Pool Fires}

Xin et al.~\cite{Xin:JSS2005} used FDS to model a 1~m diameter methane pool fire. The computational domain was 2~m by 2~m by 4~m with a uniform grid size of 2.5 cm. The predicted results were compared to  experimental data and found  to qualitatively and quantitatively reproduce  the velocity field.  The same authors performed a similar study of a 7.1~cm methane burner~\cite{Xin:CF2005} and a helium plume~\cite{Xin:CS2002}.

The 7.1 cm diameter buoyant diffusion flame has been extensively studied both experimentally and computationally. Zhou and Gore \cite{Zhou:CS1998} reported radial profiles of mixture fraction and vertical velocity for estimation of thermal expansion for natural gas buoyant diffusion flames stabilized on a 7.1 cm diameter diffuser burner. Xin et al.~\cite{Xin:CSS2002} used a Lagrangian thermal element based combustion model to simulate this flame. The authors noted that the simulations were sensitive to the burnout time utilized by the combustion model. To gain further insight into the species distribution inside the fire, Xin et al.~\cite{Xin:CF2005} performed fire dynamics simulations using a mixture fraction based combustion model. Xin and Gore \cite{Xin:CS2005} used laser-induced incandescence to determine soot distributions in vertical and horizontal planes for methane and ethane turbulent buoyant flames. Biswas et al.~\cite{Biswas:CS2007} utilized a novel time series model to simulate the scalar concentrations and temperature fields for these flames.

Hostikka et al.~\cite{Hostikka:3} modeled small pool fires of methane, natural gas and methanol to test the FDS radiation solver for low-sooting fires. They conclude that the predicted radiative fluxes are higher than measured values, especially at small heat release rates, due to an over-prediction of the gas temperature. These tests are also included in the Heat Flux section of this report.

Hietaniemi, Hostikka and Vaari~\cite{Hietaniemi:1} considered heptane pool fires of various diameters. Predictions of the burning rate as a function of diameter follow the trend observed in a number of experimental studies. Their results show an improvement in the model over the earlier work with methanol fires, due to improvements in the radiation routine and the fact that heptane is more sooty than methanol, simplifying the treatment of radiation.  The authors point out that reliable predictions of the burning rate of liquid fuels require roughly twice as fine a grid spanning the burner than would be necessary to predict plume velocities and temperatures. The reason for this is the prediction of the heat feedback to the burning surface necessary to {\em predict} rather than to {\em specify} the burning rate.


\subsection{Air and Gas Movement in the Absence of Fire}

The low Mach number assumption in FDS is appropriate not only to fire, but to most building ventilation scenarios. An example of how the model can be used to assess indoor air quality is presented by Musser et al.~\cite{Musser:1}.  The test compartment was a displacement ventilation test  room  that contained  computers, furniture, and lighting fixtures as well as heated rectangular boxes intended to represent occupants. A detailed description of the test configuration is given by Yuan et al.~\cite{Yuan:1}. The room is ventilated with cool supply air introduced via a diffuser that is mounted on a side wall near the floor. The air rises as it is warmed by heat sources and exits through a return duct located in the upper portion of the room. The  flow pattern is intended to remove contaminants by sweeping them upward at the source and removing them from the room. Sulphur hexafluoride, SF$_6$, was introduced into the compartment during the experiment as a tracer gas near the breathing zone of the  occupants.  Temperature, tracer concentration, and velocity were measured during the experiments.

In another study, Musser and Tan~\cite{Musser:2} used FDS to assess the design of ventilation systems for facilities in which train locomotives operate. Although there is only a limited amount of validation, the study is useful in demonstrating a practical use of FDS for a non-fire scenario.

Mniszewski~\cite{Mniszewski:1} used FDS to model the release of flammable gases in simple enclosures and open areas. In this work, the gases were not ignited.

Kerber and Walton~\cite{Kerber:1} provided a comparison between FDS version 1 and experiments on positive pressure ventilation in a full-scale enclosure without a fire.


\subsection{Wind Engineering}

Most applications of FDS involve fires within buildings. However, it can be used to model thermal plumes in the open and wind impinging on the exterior  of  a  building.  Rehm,  McGrattan,  Baum  and Simiu~\cite{LES:4} used the LES solver to estimate surface pressures on simple rectangular blocks in a crosswind, and compared these estimates to experimental measurements. In a subsequent paper~\cite{Rehm:WS02}, they considered the qualitative effects of multiple buildings and trees on a wind field.

A  different  approach  to  wind  was  taken  by  Wang  and Joulain~\cite{Wang:IAFSS2002}. They considered a small fire in a wind tunnel 0.4~m wide and 0.7~m tall with flow speeds of 0.5~m/s to 2.5~m/s. Much of the comparison with experiment is qualitative, including flame shape, lean, length. They also use the model to determine the predominant modes of heat transfer for different operating conditions. To assess the combustion, they implemented an ``eddy break-up'' combustion model~\cite{Magnussen:1} and compared it to the mixture fraction approach used by FDS. The two models performed better or worse, depending on the operating conditions. Some of the weaknesses of the mixture fraction model as implemented in FDS version 2 were addressed in subsequent versions.

Chang and Meroney~\cite{ChangJWE2003} compared the results of FDS with the commercial CFD package FLUENT in simulating the transport of pollutants  from steady  point sources  in an  idealized urban environment. FLUENT employs a variety of RANS (Reynolds Averaged Navier-Stokes) closure methods,  whereas FDS employs large eddy simulation (LES).  The results of the numerical models were compared with wind tunnel measurements within a 1:50 scale physical model of an urban street ``canyon.''

FDS has recently been applied to urban canopy modeling~\cite{Moon:2014} and wind engineering. Le~et~al.~\cite{Le:1997} modeled flow over a backward facing step using DNS with a Reynolds number of 5,100.  To verify the results, Jovic and Driver~\cite{JD:1994} supplemented the DNS simulation with an identically proportioned wind tunnel experiment.  Together, the data sets from these two studies have provided the baseline for analysis of recent simulations of flow over a backward facing step that are documented in this guide.

Sarwar~et~al.~\cite{Sarwar:2013} used FDS to compare SGS eddy viscosity models. The constant Smagorinsky model performed the best, although the dynamic Smagorinsky and Deardorff models, nearly equivalent in accuracy, were found to perform better than the Vreman model.  To avoid explicit specification of inlet turbulence conditions, the authors created an extremely long inlet section to allow turbulence to develop.


\subsection{Atmospheric Dispersion}

During the 1980s and 1990s, the Building and Fire Research Laboratory at NIST studied the burning of crude oil under the sponsorship of the US Minerals Management Service. The aim of the work was to assess the feasibility of using burning as a means to remove spilled oil from the sea surface. As part of the effort, Rehm and Baum developed a special application of the LES model called ALOFT. The model was a spin-off of the two-dimensional LES enclosure model, in which a three-dimensional steady-state plume was computed as a two-dimensional evolution of the lateral wind field generated by a large fire blown in a steady wind. The ALOFT model is based on large eddy simulation in that it attempts to resolve the relevant scales of a large, bent-over plume. Validation work was performed by simulating the plumes from several large experimental burns of crude oil in which aerial and ground sampling of smoke    particulate was performed~\cite{McGrattan:4a}. Yamada~\cite{ALOFT:2} performed a validation of the ALOFT model for 10~m oil tank fire. The results indicate that the prediction of the plume cross section 500~m from the fire agree well with the experimental observations.

Mouilleau and Champassith~\cite{JLP:2009} performed a validation study to assess the ability of FDS (version 4) to model atmospheric dispersion. They concluded that the best results were obtained for simulations done with explicitly-modeled wind fluctuations. Specific atmospheric flow characteristics were evaluated for passive releases in open and flat fields.



\subsection{Growing Fires}
\label{growingfires}

Floyd~\cite{Floyd:5,Floyd:6} compared FDS predictions with measurements from fire tests at the Heiss-Dampf Reaktor (HDR) facility in Germany. The structure was originally the containment building for a nuclear power reactor. The cylindrical structure was 20~m in diameter and 50~m in height topped by a hemispherical dome 10~m in radius.  The building was divided into eight levels. The total volume of the building was approximately 11,000~m$^3$. From 1984 to 1991, four fire test series were performed within the HDR facility. The T51 test series consisted of eleven propane gas tests and three wood crib tests.

FDS predictions of fire growth and smoke movement in large spaces were performed by Kashef~\cite{Kashef:1}. The experiments were conducted at the National Research Council Canada. The tests were performed in a compartment with dimensions of 9~m by 6~m by 5.5~m with 32 exhaust inlets and a single supply fan. A burner generated fires ranging in size from 15~kW to 1000~kW.


\subsection{Flame Spread}
\label{flame spread}

Although FDS simulations have been compared to actual and experimental large-scale fires, it is difficult to quantify the accuracy because of the uncertainty associated with material properties. Most quantified validation work associated with flame spread have been for small, laminar flames with length scales ranging from millimeters to a few centimeters.

For example, FDS (or its core algorithms) have been used at a grid resolution of roughly 1~mm to look at flames spreading over paper in a micro-gravity environment~\cite{McGrattan:CF1996,Kashiwagi:CS1996,Mell:CS98,Mell:CS00,Prasad:CS2002,Nakamura:CF2002}, as well as ``g-jitter'' effects aboard spacecraft~\cite{Mell:g-jitter}. Simulations have been compared to experiments performed aboard the Space Shuttle. The flames are laminar and relatively simple in structure, and the materials are relatively well-characterized.

FDS flame spread predictions were compared to experiments over a 5~m slab of PMMA performed by Factory Mutual Research Corporation (FMRC)~\cite{Ma:2,Ma:3}.

A  charring model  was  implemented in  FDS  by Hostikka  and McGrattan~\cite{Hostikka:2}. The model was a simplification of work done at NIST by Ritchie et al.~\cite{Ritchie:1}. The charring model was first used to predict the burning rate of a small wooden sample in the cone calorimeter. Full-scale room tests with wood paneling were modeled, but the results were grid-dependent. This was likely a consequence of the gas phase spatial resolution, rather than the solid phase.

Kwon et al.~\cite{Kwon:Fire_Technology_2007} performed three simulations to evaluate the capability of FDS, version 4, in predicting upward flame spread. The FDS predictions were compared with empirical correlations and experimental data for upward flame spread on a 5~m PMMA panel. A simplified flame spread model was also applied to assess the simulation results.

An extensive amount of flame spread validation work with FDS version 4 has been performed  by  Hietaniemi,  Hostikka,  and  Vaari  at  VTT, Finland~\cite{Hietaniemi:1}. The case studies are comprised of fire experiments ranging in scale from the cone calorimeter (ISO~5660-1, 2002) to full-scale fire tests such as the room corner test (ISO~9705, 1993). Comparisons are also made between FDS 4 results and data obtained in the SBI (Single Burning Item) Euro-classification test apparatus (EN 13823, 2002) as well as data obtained in two ad hoc experimental configurations: one is similar to the room corner test but has only partial linings and the other is a space to study fires in building cavities. In the study of upholstered furniture, the experimental configurations are the cone and furniture calorimeters, and the ISO room. For liquid pool fires, comparison is made to data obtained by numerous researchers.  The burning materials include spruce timber, MDF (Medium Density Fiber) board, PVC wall carpet, upholstered furniture, cables with plastic sheathing, and heptane. The scope of the VTT work is considerable. Assessing the accuracy of the model must be done on a case by case basis. In some cases, predictions of the burning rate of the material were based solely on its fundamental properties, as in the heptane pool fire simulations. In other cases, some properties of the material are unknown, as in the spruce timber simulations. Thus, some of the simulations are true predictions, some are calibrations. The intent of the authors was to provide guidance to engineers using the model as to appropriate grid sizes and material properties. In many cases, the numerical grid was made fairly coarse to account for the fact that in practice, FDS is used to model large spaces of which the fuel may only comprise a small fraction.

Mangs and Hostikka~\cite{Mangs_Hostikka:IAFSS10} carried out experiments and simulations (FDS 5.4.3) of the vertical flame spread on the surface of thin birch wood cylinders at different ambient temperatures. The parameters for the pyrolysis model were estimated from TGA and cone calorimeter experiments. The gas phase flow was calculated in the DNS mode with 1.0 mm grid cells in axi-symmetric geometry. The simulation model was able to predict the flame spread rates within the uncertainties associated with the experiments and postsimulation analysis of the spread rate.


\subsection{Compartment Fires}

As part of the NIST investigation of the World Trade Center fires and collapse, a series of large scale fire experiments were performed
specifically to validate FDS~\cite{NIST_NCSTAR_1-5B}.  The tests were performed in a rectangular compartment 7.2~m long by 3.6~m wide by 3.8~m
tall.  The fires were fueled by heptane for some tests and a heptane/toluene mixture for the others. The results of the experiments and simulations
are included in detail in this Guide.

A second set of experiments to validate FDS for use in the World Trade Center investigation is documented in Ref.~\cite{NIST_NCSTAR_1-5E}. The experiments
are not described as part of this Guide. The intent
of these tests was to evaluate the ability of the model to simulate the growth of a fire burning three office workstations within a compartment of
dimensions 11~m by 7~m by 4~m, open at one end to mimic the ventilation of windows similar to those in the WTC towers. Six tests were performed
with various initial conditions exploring the effect of jet fuel spray and ceiling tiles covering the surface of the desks and carpet. Measurements
were made of the heat release rate and compartment gas  temperatures at four  locations using vertical thermocouple arrays. Six different
material samples were tested in the NIST cone calorimeter: desk, chair, paper, computer case, privacy panel, and carpet. Data for the carpet,
desk and privacy panel were input directly into FDS, with the other three materials lumped together to form an idealized fuel type. Open burns of
single workstations were used to calibrate the simplified fuel package. Details of the modeling are contained in Ref.~\cite{NIST_NCSTAR_1-5F}.


The BRE Centre for Fire Safety Engineering at the University of Edinburgh conducted a series of large-scale fire tests in a real high rise building in Dalmarnock, Glasgow,
Scotland~\cite{Rein:Dalmarnock,Rein:FSJ}.
The experiments took place in July, 2006, with the close collaboration of the Strathclyde Fire Brigade and other partners.
These experiments attempted to create realistic scenario in which a wide range of modern fire safety engineering tools could be put to a test.
Jahn, Rein and Torero assessed the sensitivity of FDS when applied to these experiments~\cite{Jahn:IAFSS9}. Fire size and
location, convection, radiation and combustion parameters were varied in order to determine the associated
degree of sensitivity. Emphasis was put in the prediction of secondary ignition and time to flashover. In this
context and while keeping the HRR constant, simulations of fire growth were significantly sensitive to
location of the heat release rate, fire area, flame radiative fraction, and material thermal and ignition
properties.

Students at Stord/Haugesund University College in Norway simulated full-scale experiments of temperature and smoke spread in a
realistic multi-room setting using both CFAST and FDS~\cite{Storm:thesis}. Data from the top 0.5~m of the compartments was compared with measurements.
The simulations were found to provide satisfying results in CFAST, as an alternative to FDS.

Drean et al.~\cite{Drean2018} measured the fire exposure to an exterior facade of a two-level test facility operated by the engineering firm Efectis in Saint-Aubin, France. The objective was to evaluate the ability of FDS, version~6, to predict the gas temperatures and heat fluxes at the exterior wall.

Chaudhari et al.~\cite{Chaudhari:FT2023} simulated controlled fire experiments using a gas burner conducted in a purpose-built, two-story, moderately air-tight residential structure. Temperatures, gas concentrations (oxygen, water vapor, carbon dioxide), and differential pressures were monitored throughout the structure. HVAC status and stairwell door openings to the fire room were varied for the experiments.



\subsection{Sprinklers, Mist System, and Suppression by Water}

Vettori~\cite{Vettori:1} modeled sprinkler activation patterns in a room with an obstructed ceiling. In a follow-up report, Vettori~\cite{Vettori:2} extended his study to include sloped ceilings, with and without obstructions. Both of these experimental series are included within the current validation guide and are referred to the Vettori Flat and Sloped Ceiling Experiments.

A  significant validation effort  for sprinkler  activation and suppression was a project entitled the International Fire Sprinkler, Smoke and Heat Vent, Draft Curtain Fire Test Project organized by the National Fire Protection  Research Foundation~\cite{McGrattan:5}. Thirty-nine large scale fire tests were conducted at Underwriters Laboratories in Northbrook, IL. The tests were aimed at evaluating the performance of various fire protection systems in large buildings with flat ceilings, like warehouses and ``big box'' retail stores. All the tests were conducted under a 30~m by 30~m adjustable-height platform in a 37~m by 37~m by 15~m high test bay. At the time, FDS had not been publicly released and was referred to as the Industrial Fire Simulator (IFS), but it was essentially the same as FDS version 1. The first and second series of heptane spray burner fires are included in this guide under the heading ``UL/NFPRF Sprinkler, Vent, and Draft Curtain Study.'' Most of the full-scale experiments performed during the project used a heptane spray burner to  generate controlled fires of 1~MW to 10~MW. However, five experiments were performed with 6~m high racks containing the Factory Mutual Standard Plastic Commodity, or Group A Plastic. To model these fires, bench scale experiments were performed to characterize the burning behavior of the commodity, and larger test fires provided validation data with which to test the model predictions of the burning rate and flame spread behavior~\cite{Hamins:1,Hamins:IAFSS2002}. Two to four tier configurations were evaluated.

High rack storage fires of pool chemicals were modeled by Olenick et al.~\cite{Olenick:1} to determine the validity of sprinkler activation predictions of FDS. The model was compared to full-scale fires conducted in January, 2000 at Southwest Research Institute in San Antonio, Texas.

FDS has been used to study the behavior of a fire undergoing suppression by a water mist system. Kim and Ryou~\cite{Kim:BE2003,Kim:IJACR2004}  compared FDS  predictions to results of compartment fire tests with and without the application of a water mist. The cooling and oxygen dilution were predicted to within about 10~\% of the measurements, but the simulations failed to predict the complete extinguishment of a hexane pool fire. The authors suggest that this is a result of the combustion model rather than the spray or droplet model.

Another study of water mist suppression using FDS was conducted by Hume  at  the  University  of  Canterbury,  Christchurch,  New Zealand~\cite{Hume:Masters}. Full-scale experiments were performed in which a fine water mist was combined with a displacement ventilation system to protect occupants and electrical equipment in the event of a fire. Simulations of these experiments with FDS showed qualitative agreement, but the version of the model used in the study (version 3) was not able to predict accurately the decrease in heat release rate of the fire.

Hostikka  and  McGrattan~\cite{Hostikka:FSJ2006}  evaluated  the absorption of thermal radiation by water sprays. They considered two sets of experimental data and concluded that FDS has the ability to predict the attenuation of thermal radiation ``when the hydrodynamic interaction between  the droplets is  weak.'' However, modeling interacting sprays would require a more costly coalescence model. They also note that the results of the model were sensitive to grid size, angular discretization, and droplet sampling.

O'Grady and Novozhilov~\cite{OGrady:CST} compared the predictions of FDS version 4 against full-scale fire tests performed at SP Sweden involving a 1.5~MW steady-state fire with two different sprinkler flow rates~\cite{Ingason:1}. The authors reported results for gas temperatures and the tangential flow velocity in the ceiling jet. Sensitivity of the model to a range of input parameters was investigated. The model demonstrated moderate sensitivity to the spray parameters, such as spray cone configuration, initial droplet velocities, and droplet sizes. On the other hand, the sensitivity to other parameters such as sprinkler atomization length and rms of droplet size distribution was low.

Xiao~\cite{Xiao:FT2012} compared FDS simulations with real scale compartment measurements for unsprinklered and sprinklered experiments. Numerical results for doorway mass flow rate and temperature are compared with the experimental data for three fire sizes.



\subsection{Airflows in Fire Compartments}

Friday and Mowrer~\cite{Friday:1} studied the use of FDS in large scale mechanically ventilated spaces.  The ventilated enclosure was provided with air injection rates of 1 to 12 air changes per hour and a fire with heat release rates ranging from 0.5~MW to 2~MW. The test measurements and model output were compared to assess the accuracy of FDS. These simulations have been repeated with the latest version of FDS and reported in this guide under the heading ``FM/SNL Test Series.''

Zhang et al.~\cite{Zhang:2} utilized the FDS model to predict turbulence characteristics of the flow and temperature fields due to fire in a compartment.  The experimental data was acquired through tests that replicated a half-scale ISO Room Fire Test. Two cases were explored -- the heat source in the center of the room and the heat source adjacent to a wall. In both cases, the heat source was a heating element with an output of 12~kW/m$^2$.


\subsection{Tunnel Fires}

Cochard~\cite{Cochard:1} used FDS to study the ventilation within a tunnel. He compared the model results with a full-scale tunnel fire experiment conducted as part of the Massachusetts Highway Department Memorial Tunnel Fire Ventilation Test Program. The test consisted of a single point supply of fresh air through a 28~m$^2$ opening in a 135~m tunnel.

McGrattan and Hamins~\cite{McGrattan:HST} also applied FDS to simulate two of the Memorial Tunnel Fire Tests as validation for the use of the model in studying an actual fire in the Howard Street Tunnel, Baltimore, Maryland, July 2001. The experiments chosen for the comparison were unventilated. One experiment was a 20~MW fire; the other a 50~MW fire.

Piergoirgio et al.~\cite{Piergiorgio:1} provided a qualitative analysis of FDS applied to a truck fire within a tunnel. The goal of their analysis was to describe the spread of the toxic gases within the tunnels, to determine the places not involved in the spreading of combustion products and to quantify the oxygen, carbon monoxide and hydrochloric acid concentrations during the fire.

Edwards et al.~\cite{Edwards:SME2005,Edwards:FSJ} used FDS to determine the critical air velocity for smoke reversal in a tunnel as a function of the fire intensity, and his results compared favorably with  experimental results.  In a  further study,  Edwards and Hwang~\cite{Edwards:SME2006} applied FDS to study fire spread along combustibles in a ventilated mine entry. Analyses such as these are intended for planning and implementation of ventilation changes during mine fire fighting and rescue operations.

Bilson et al.~\cite{Bilson:2008} used FDS to evaluate the interaction of a deluge system with a tunnel ventilation and smoke exhaust system.

Harris~\cite{Harris:ISTSS2010} used FDS to determine the heat flux from a tunnel fire under varying water application rates. These results were qualitatively consistent with experimental results of Arvidson~\cite{Arvidson:ISTSS2010}, who conducted burn tests for shielded and unshielded standard plastic commodities under a variety of spray conditions.

Trelles and Mawhinney~\cite{Trelles:JFPE2010,Mawhinney:FT2012} simulated with FDS~4 a series of full-scale fire suppression experiments conducted at the San Pedro de Anes test tunnel near Gijon, Asturias, Spain in February, 2006. The fuel consisted of wooden and polyethylene pallets, and the suppression system consisted of different configurations of water mist nozzles.




\subsection{Smoke Detection}

The ability of version 1 of FDS to accurately predict smoke detector activation was studied by D'Souza~\cite{DSouza:1}. The smoke transport model within FDS was tested and compared with UL~217 test data. The second step in this research was to further validate the model with full-scale multi-compartment fire tests. The results indicated that FDS is capable of predicting smoke detector activation when used with smoke detector lag correlations that correct for the time delay associated with smoke having to penetrate the detector housing. A follow-up report by Roby et al.~\cite{CSE_GCR} and paper by Zhang et al.~\cite{Zhang:FSJ2008} describes the implementation and validation of the smoke detector algorithm currently incorporated in FDS.

Another study of smoke detector activation was carried out by Brammer at the University of Canterbury, New Zealand~\cite{Brammer:1}. Two fire tests from a series performed in a two-story residence were simulated, and smoke detector activation times were predicted using three different methods. The methods consisted of either a temperature correlation, a time-lagged function of the optical density, or a thermal device much like a heat detector. The purpose was to identify ways to reliably predict smoke detector activation using typical model output like temperature and smoke concentration. It was remarked that simulating the early stage of the fire is critical to reliable prediction.

Cleary~\cite{Cleary:1} also provided a comparison between FDS computed gas velocity, temperature and concentrations at various detector locations.  The research concluded that multi-room fire simulations with the FDS model can accurately predict the conditions that a sensor might experience during a real fire event.




\subsection{Combustion Model}

A few studies have been performed comparing direct numerical simulations (DNS) of a simple burner flame to laboratory experiments~\cite{Mukhopadhyay:1}. Another study compared DNS calculations of a counterflow diffusion flames to experimental measurements and the results of a one-dimensional multi-step kinetics model~\cite{Hamins:NASA}.

Bundy, Dillon and Hamins~\cite{Dillon:1,Hamins:FPE2005} studied the use of FDS in providing data and correlations for fire investigators to support their investigations.  A paraffin wax candle was placed within a small plexi-glass enclosure. The heat flux from the candle flame was modeled with FDS.

Floyd et al.~\cite{Floyd:1,Floyd:6} compared the radiation model of FDS version 2 with full-scale data from the Virginia Tech Fire Research Laboratory (VTFRL). The test compartment was outfitted with equipment capable  of taking  temperature,  air velocity,  gas concentrations, unburned hydrocarbon and heat flux measurements. The test facility consisted of a single compartment geometrically similar to the ISO 9705 standard compartment with dimensions of 1.2~m by 1.8~m by 1.2~m in height.  The ceiling and walls were constructed of fiberboard over a steel shell with a floor of concrete.  Three baseline experiments were completed with fires ranging in size from 90~kW to 440~kW.

Xin  and Gore~\cite{Xin:JSS2003}  compared  FDS predictions  and measurements of the spectral radiation intensities of small fires. The fuel flow rates for methane and ethylene burners were selected so that the Froude numbers matched that of liquid toluene pool fires. The heat release rate was 4.2~kW for the methane flame and 3.4~kW for the ethylene flame. Line of sight spectral radiation intensities were measured at six downstream locations.  The spectral radiation intensity calculations were performed by post-processing the transient scalar distributions provided by FDS.

Zhang et al.~\cite{Zhang:1} compared the experimental results of a circular methane gas burner to predictions computed by FDS. The compartment was 2.8~m by 2.8~m by 2.2~m high with natural ventilation from a standard door.




\subsection{Soot Deposition}

Several studies have been conducted that indicate soot deposition is an important factor in compartment fires
for the accurate prediction of smoke concentrations, smoke detector activations, and visibility.
Gottuk et al.~\cite{Gottuk:IAFSS2008} reported that smoke concentrations predicted by FDS near smoke alarms in a
corridor were two to five times greater than measured smoke concentrations. Hamins et al.~\cite{Hamins:SP1013-1}
conducted full-scale compartment fire experiments for use in validation studies of various fire models, including FDS.
The results indicated that smoke concentrations predicted by FDS were up to five times greater than measured
smoke concentrations. Floyd and McDermott~\cite{Floyd:Interflam2010} implemented thermophoretic and turbulent diffusion soot
deposition mechanisms in FDS and compared predicted soot densities and concentrations to measurements from small- and large-scale
experiments. Riahi~\cite{Riahi:2011} conducted bench-scale experiments to measure soot densities and soot
deposition patterns on walls for various fuels. Riahi identified thermophoretic deposition as an important soot deposition
mechanism in the hot gas layer. Cohan~\cite{Cohan:Masters} used FDS to simulate select cases from the
Gottuk~\cite{Gottuk:IAFSS2008} corridor tests, Hamins et al.~\cite{Hamins:SP1013-1}~NRC experiments, and
Riahi~\cite{Riahi:2011} hood experiments with thermophoretic and turbulent diffusion soot deposition mechanisms.
Overholt~and~Ezekoye~\cite{Overholt:1} implemented gravitational settling of soot in the gas-phase in FDS and quantified
the effects of gravitational settling/deposition compared to thermophoretic and turbulent diffusion deposition for small- and
large-scale validation cases.




\section{Reconstructions of Actual Fires}

ASTM E 1355 states that a model may be evaluated by comparing it with ``Documented Fire Experience'' which includes:
\begin{itemize}
\item eyewitness accounts of real fires,
\item known behavior of materials in fires (for example, melting temperatures of materials), and
\item observed post-fire conditions, such as the extent of fire spread.
\end{itemize}
Often the term ``reconstruction'' is applied to this type of simulation, because the model is used to reconstruct events based on evidence collected during and after the fire. Some of the more notable studies performed at NIST include:
\begin{itemize}
\item McGrattan, Bouldin, and Forney simulated the fires within the World Trade Center towers and Building 7 on September 11, 2001~\cite{NIST_NCSTAR_1-5F}.
\item Grosshandler et al.~investigated the fire that occurred at the Station Nightclub in Rhode Island in February, 2003~\cite{Grosshandler:Station}.
\item Madrzykowski and Vettori examined a fire in a townhouse in Washington, D.C., where two fire fighters were killed and one severely injured in 1999~\cite{Madrzykowski:1}.
\item Vettori, Madrzykowski, and Walton simulated a fire in a Houston restaurant that killed two fire fighters in 2000~\cite{Texas}.
\item Madrzykowski, Forney and Walton simulated a fire that killed three children and three fire fighters in a two story duplex house in Iowa in 1999~\cite{Iowa}.
\item Madrzykowski and Walton investigated the fire in the Cook County (Chicago) Administration Building in October, 2003, that killed six people trapped in a stairwell~\cite{Cook_County}.
\item Bryner~et~al.~simulated a fire in a large furniture store that occurred in June, 2007, killing nine fire fighters~\cite{Bryner:Charleston}.
\end{itemize}
Outside of NIST, FDS has been used to investigate many actual fires, but very few of these studies are documented in the literature. Exceptions
include:
\begin{itemize}
\item A large fire in a ``cash \& carry'' warehouse in the UK was studied by Camp and Townsend using both hand calculations and FDS (version 1)~\cite{Camp:Interflam2001}.
\item A study by Rein et al.~\cite{Rein:Interflam2004} looked at several fire events using an analytical fire growth model, the NIST zone model CFAST, and FDS.
\item A similar study was performed several years earlier by Spearpoint et al.~\cite{Spearpoint:ICFRE3} as a class exercise at the University of Maryland.
\item During the SFPE Professional Development Week in the fall of 2001, a workshop was held in which several engineers related their experiences using FDS as a forensic tool~\cite{Carpenter:SFPE2001}.
\item The role of carbon monoxide in the deaths of three fire fighters was studied by Christensen and Icove~\cite{Christensen:JFS}.
\end{itemize}
