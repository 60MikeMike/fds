% !TEX root = FDS_Validation_Guide.tex

\chapter{Fire Plumes}

\section{Plume Temperatures}

For fire plumes, a measure of how well the flow field is resolved is given by the non-dimensional expression $D^*/\dx$, where $D^*$ is a characteristic
fire diameter
\be
   D^* = \left(
     \frac{\dQ}{\rho_\infty \, c_p \, T_\infty \, \sqrt{g} }
     \right)^\frac{2}{5}
\ee
and $\dx$ is the nominal size of a mesh cell\footnote{The characteristic fire diameter is related to the characteristic fire size via the relation $Q^* = (D^*/D)^{5/2}$, where $D$ is the physical diameter of the fire.}. The quantity $D^*/\dx$ can be thought of as the number of computational cells spanning the characteristic (not necessarily the physical) diameter of the fire. The more cells spanning the fire, the better the resolution of the calculation. It is better to assess the quality of the mesh in terms of this non-dimensional parameter, rather than an absolute mesh cell size. For example, a cell size of 10~cm may be ``adequate,'' in some sense, for evaluating the spread of smoke and heat through a building from a sizable fire, but may not be appropriate to study a very small, smoldering source. The resolution of all the numerical simulations included in this chapter is given in Table~\ref{Numerical_Parameters}.

\clearpage

\subsection{FM Burner Experiments}
\label{FM_Burner_Plume}

Mean and rms temperature measurements were made above a 13.7~cm (inner) diameter, 15~kW ethylene burner. The radial profiles are located at heights of 1.0, 1.5, 2.0, 2.5, 3.0, and 3.5 burner diameters, $D$. Figure~\ref{FM_Burner_Temperature_PDF} displays the probability distributions (PDFs) at the six heights and radii of 0~cm, 1~cm, 2~cm, 3~cm, and 4~cm.

\begin{figure}[!h]
\begin{tabular*}{\textwidth}{l@{\extracolsep{\fill}}r}
\includegraphics[height=2.15in]{SCRIPT_FIGURES/FM_Burner/Radial_Mean_Temperature_1p0D} &
\includegraphics[height=2.15in]{SCRIPT_FIGURES/FM_Burner/Radial_RMS_Temperature_1p0D} \\
\includegraphics[height=2.15in]{SCRIPT_FIGURES/FM_Burner/Radial_Mean_Temperature_1p5D} &
\includegraphics[height=2.15in]{SCRIPT_FIGURES/FM_Burner/Radial_RMS_Temperature_1p5D} \\
\includegraphics[height=2.15in]{SCRIPT_FIGURES/FM_Burner/Radial_Mean_Temperature_2p0D} &
\includegraphics[height=2.15in]{SCRIPT_FIGURES/FM_Burner/Radial_RMS_Temperature_2p0D}
\end{tabular*}
\caption[FM Burner experiments, plume mean and rms temperatures]
{FM Burner experiments, plume mean and rms temperatures at heights of 1.0, 1.5, and 2.0 diameters.}
\label{FM_Burner_Plume_1}
\end{figure}

\begin{figure}[p]
\begin{tabular*}{\textwidth}{l@{\extracolsep{\fill}}r}
\includegraphics[height=2.15in]{SCRIPT_FIGURES/FM_Burner/Radial_Mean_Temperature_2p5D} &
\includegraphics[height=2.15in]{SCRIPT_FIGURES/FM_Burner/Radial_RMS_Temperature_2p5D} \\
\includegraphics[height=2.15in]{SCRIPT_FIGURES/FM_Burner/Radial_Mean_Temperature_3p0D} &
\includegraphics[height=2.15in]{SCRIPT_FIGURES/FM_Burner/Radial_RMS_Temperature_3p0D} \\
\includegraphics[height=2.15in]{SCRIPT_FIGURES/FM_Burner/Radial_Mean_Temperature_3p5D} &
\includegraphics[height=2.15in]{SCRIPT_FIGURES/FM_Burner/Radial_RMS_Temperature_3p0D}
\end{tabular*}
\caption[FM Burner experiments, plume mean and rms temperatures]
{FM Burner experiments, plume mean and rms temperatures at heights of 2.5, 3.0, and 2.5 diameters.}
\label{FM_Burner_Plume_2}
\end{figure}

\begin{sidewaysfigure}[p]
\begin{tabular*}{\textwidth}{ccccc}
\includegraphics[width=1.5in]{SCRIPT_FIGURES/FM_Burner/Temperature_PDF_1p0D_r=0cm} &
\includegraphics[width=1.5in]{SCRIPT_FIGURES/FM_Burner/Temperature_PDF_1p0D_r=1cm} &
\includegraphics[width=1.5in]{SCRIPT_FIGURES/FM_Burner/Temperature_PDF_1p0D_r=2cm} &
\includegraphics[width=1.5in]{SCRIPT_FIGURES/FM_Burner/Temperature_PDF_1p0D_r=3cm} &
\includegraphics[width=1.5in]{SCRIPT_FIGURES/FM_Burner/Temperature_PDF_1p0D_r=4cm} \\
\includegraphics[width=1.5in]{SCRIPT_FIGURES/FM_Burner/Temperature_PDF_1p5D_r=0cm} &
\includegraphics[width=1.5in]{SCRIPT_FIGURES/FM_Burner/Temperature_PDF_1p5D_r=1cm} &
\includegraphics[width=1.5in]{SCRIPT_FIGURES/FM_Burner/Temperature_PDF_1p5D_r=2cm} &
\includegraphics[width=1.5in]{SCRIPT_FIGURES/FM_Burner/Temperature_PDF_1p5D_r=3cm} &
\includegraphics[width=1.5in]{SCRIPT_FIGURES/FM_Burner/Temperature_PDF_1p5D_r=4cm} \\
\includegraphics[width=1.5in]{SCRIPT_FIGURES/FM_Burner/Temperature_PDF_2p0D_r=0cm} &
\includegraphics[width=1.5in]{SCRIPT_FIGURES/FM_Burner/Temperature_PDF_2p0D_r=1cm} &
\includegraphics[width=1.5in]{SCRIPT_FIGURES/FM_Burner/Temperature_PDF_2p0D_r=2cm} &
\includegraphics[width=1.5in]{SCRIPT_FIGURES/FM_Burner/Temperature_PDF_2p0D_r=3cm} &
\includegraphics[width=1.5in]{SCRIPT_FIGURES/FM_Burner/Temperature_PDF_2p0D_r=4cm} \\
\includegraphics[width=1.5in]{SCRIPT_FIGURES/FM_Burner/Temperature_PDF_2p5D_r=0cm} &
\includegraphics[width=1.5in]{SCRIPT_FIGURES/FM_Burner/Temperature_PDF_2p5D_r=1cm} &
\includegraphics[width=1.5in]{SCRIPT_FIGURES/FM_Burner/Temperature_PDF_2p5D_r=2cm} &
\includegraphics[width=1.5in]{SCRIPT_FIGURES/FM_Burner/Temperature_PDF_2p5D_r=3cm} &
\includegraphics[width=1.5in]{SCRIPT_FIGURES/FM_Burner/Temperature_PDF_2p5D_r=4cm} \\
\includegraphics[width=1.5in]{SCRIPT_FIGURES/FM_Burner/Temperature_PDF_3p0D_r=0cm} &
\includegraphics[width=1.5in]{SCRIPT_FIGURES/FM_Burner/Temperature_PDF_3p0D_r=1cm} &
\includegraphics[width=1.5in]{SCRIPT_FIGURES/FM_Burner/Temperature_PDF_3p0D_r=2cm} &
\includegraphics[width=1.5in]{SCRIPT_FIGURES/FM_Burner/Temperature_PDF_3p0D_r=3cm} &
\includegraphics[width=1.5in]{SCRIPT_FIGURES/FM_Burner/Temperature_PDF_3p0D_r=4cm} \\
\includegraphics[width=1.5in]{SCRIPT_FIGURES/FM_Burner/Temperature_PDF_3p5D_r=0cm} &
\includegraphics[width=1.5in]{SCRIPT_FIGURES/FM_Burner/Temperature_PDF_3p5D_r=1cm} &
\includegraphics[width=1.5in]{SCRIPT_FIGURES/FM_Burner/Temperature_PDF_3p5D_r=2cm} &
\includegraphics[width=1.5in]{SCRIPT_FIGURES/FM_Burner/Temperature_PDF_3p5D_r=3cm} &
\includegraphics[width=1.5in]{SCRIPT_FIGURES/FM_Burner/Temperature_PDF_3p5D_r=4cm}
\end{tabular*}
\caption[FM Burner experiments, temperature PDFs]
{FM Burner experiments, temperature PDFs.}
\label{FM_Burner_Temperature_PDF}
\end{sidewaysfigure}


\clearpage

\subsection{FM/SNL Experiments}

\label{FM/SNL_Plume}

The FM/SNL tests consisted of propylene gas burners, heptane pools, methanol pools, PMMA solids, as well as qualified and unqualified cables, burned in a large room which, for the first 18 tests, was free of obstructions. Plume Temperatures shown here were measured at approximately 6~m from the floor, or 0.98 times the total ceiling height. For Tests~1-5 and 7-9, the thermocouple station (Station~13) was centered above the fire pan. Tests~6 and 10-15 used an alternate fire location, centered along the south wall. Station~9 was not centered above these fires, but fell within the plume. Tests~16 and 17 had fires located in the south-west corner of the room, too remote from any stations to allow for plume measurements.

\begin{figure}[!h]
\begin{tabular*}{\textwidth}{l@{\extracolsep{\fill}}r}
\includegraphics[height=2.15in]{SCRIPT_FIGURES/FM_SNL/FM_SNL_01_Plume_Temperature} &
\includegraphics[height=2.15in]{SCRIPT_FIGURES/FM_SNL/FM_SNL_02_Plume_Temperature} \\
\includegraphics[height=2.15in]{SCRIPT_FIGURES/FM_SNL/FM_SNL_03_Plume_Temperature} &
\includegraphics[height=2.15in]{SCRIPT_FIGURES/FM_SNL/FM_SNL_04_Plume_Temperature}
\end{tabular*}
\caption[FM/SNL experiments, plume temperature, Tests 1-4]
{FM/SNL experiments, plume temperature, Tests 1-4.}
\label{FM_SNL_Plume_1}
\end{figure}

\newpage

\begin{figure}[p]
\begin{tabular*}{\textwidth}{l@{\extracolsep{\fill}}r}
\includegraphics[height=2.15in]{SCRIPT_FIGURES/FM_SNL/FM_SNL_05_Plume_Temperature} &
\includegraphics[height=2.15in]{SCRIPT_FIGURES/FM_SNL/FM_SNL_06_Plume_Temperature} \\
\includegraphics[height=2.15in]{SCRIPT_FIGURES/FM_SNL/FM_SNL_07_Plume_Temperature} &
\includegraphics[height=2.15in]{SCRIPT_FIGURES/FM_SNL/FM_SNL_08_Plume_Temperature} \\
\includegraphics[height=2.15in]{SCRIPT_FIGURES/FM_SNL/FM_SNL_09_Plume_Temperature} &
\includegraphics[height=2.15in]{SCRIPT_FIGURES/FM_SNL/FM_SNL_10_Plume_Temperature} \\
\includegraphics[height=2.15in]{SCRIPT_FIGURES/FM_SNL/FM_SNL_11_Plume_Temperature} &
\includegraphics[height=2.15in]{SCRIPT_FIGURES/FM_SNL/FM_SNL_12_Plume_Temperature}
\end{tabular*}
\caption[FM/SNL experiments, plume temperature, Tests 5-12]
{FM/SNL experiments, plume temperature, Tests 5-12.}
\label{FM_SNL_Plume_2}
\end{figure}

\begin{figure}[p]
\begin{tabular*}{\textwidth}{l@{\extracolsep{\fill}}r}
\includegraphics[height=2.15in]{SCRIPT_FIGURES/FM_SNL/FM_SNL_13_Plume_Temperature} &
\includegraphics[height=2.15in]{SCRIPT_FIGURES/FM_SNL/FM_SNL_14_Plume_Temperature} \\
\includegraphics[height=2.15in]{SCRIPT_FIGURES/FM_SNL/FM_SNL_15_Plume_Temperature} &
\includegraphics[height=2.15in]{SCRIPT_FIGURES/FM_SNL/FM_SNL_16_Plume_Temperature} \\
\includegraphics[height=2.15in]{SCRIPT_FIGURES/FM_SNL/FM_SNL_17_Plume_Temperature} &
\includegraphics[height=2.15in]{SCRIPT_FIGURES/FM_SNL/FM_SNL_21_Plume_Temperature} \\
\includegraphics[height=2.15in]{SCRIPT_FIGURES/FM_SNL/FM_SNL_22_Plume_Temperature} &
\end{tabular*}
\caption[FM/SNL experiments, plume temperature, Tests 13-17, 21-22]
{FM/SNL experiments, plume temperature, Tests 13-17, 21-22.}
\label{FM_SNL_Plume_3}
\end{figure}

\clearpage



\subsection{McCaffrey's Plume Correlation}

The following plots show the results of simulations of McCaffrey's five fires at three grid resolutions, nominally $D^*/\dx=[5,10,20]$ (respectively, coarse, medium, and fine resolution).  Temperature measurements and reported simulation results are for thermocouple temperature.

\begin{figure}[h!]
\begin{tabular*}{\textwidth}{l@{\extracolsep{\fill}}r}
\includegraphics[height=2.15in]{SCRIPT_FIGURES/McCaffrey_Plume/McCaffrey_Plume_Temperature_14_kW} &
\includegraphics[height=2.15in]{SCRIPT_FIGURES/McCaffrey_Plume/McCaffrey_Plume_Temperature_22_kW} \\
\includegraphics[height=2.15in]{SCRIPT_FIGURES/McCaffrey_Plume/McCaffrey_Plume_Temperature_33_kW} &
\includegraphics[height=2.15in]{SCRIPT_FIGURES/McCaffrey_Plume/McCaffrey_Plume_Temperature_45_kW} \\
\multicolumn{2}{c}{\includegraphics[height=2.15in]{SCRIPT_FIGURES/McCaffrey_Plume/McCaffrey_Plume_Temperature_57_kW}}
\end{tabular*}
\caption[McCaffrey experiments, plume temperature]
{McCaffrey experiments, plume temperature.}
\label{McCaffrey_Plume_Temperature}
\end{figure}

\clearpage

Below we plot the same results but arranged in a different way.  The height dimension is scaled by the fire Froude number and each plot represents nominally the same resolution level.

\begin{figure}[h!]
\begin{tabular*}{\textwidth}{l@{\extracolsep{\fill}}r}
\includegraphics[height=2.15in]{SCRIPT_FIGURES/McCaffrey_Plume/McCaffrey_Temperature_Correlation_Coarse} &
\includegraphics[height=2.15in]{SCRIPT_FIGURES/McCaffrey_Plume/McCaffrey_Temperature_Correlation_Medium} \\
\multicolumn{2}{c}{\includegraphics[height=2.15in]{SCRIPT_FIGURES/McCaffrey_Plume/McCaffrey_Temperature_Correlation_Fine}}
\end{tabular*}
\caption[McCaffrey experiments, plume temperature, Froude scaling]
{McCaffrey experiments, plume temperature, Froude scaling.}
\label{McCaffrey_Plume_Temperature_Froude}
\end{figure}

\clearpage

\subsection{NIST/NRC Corner Effects Experiments}

This set of experiments involved a 60~cm by 60~cm natural gas burner with heat release rates of 200~kW, 300~kW, and 400~kW. The burner was initially positioned in a corner or against a wall and then gradually moved away. A three-tiered array of thermocouples was positioned above the burner and moved along with it. Each tier contained 29~bare-bead thermocouples at heights of 2.13~m, 2.74~m, and 3.35~m above the floor. The plots below show the maximum temperature of the 29~thermocouples, time-averaged over 2~min, at each level. Note that FDS does not allow its devices to move; thus the time over which the model and measurement are compared is limited to that time period in the experiment when the center of the burner was underneath the fixed TC array in the corner or against the wall. In short, the last 30~min of the experiments are not considered.

\begin{figure}[!h]
\begin{tabular*}{\textwidth}{l@{\extracolsep{\fill}}r}
\includegraphics[height=2.15in]{SCRIPT_FIGURES/NIST_NRC_Corner_Effects/corner_200_kW_Corner_Plume} &
\includegraphics[height=2.15in]{SCRIPT_FIGURES/NIST_NRC_Corner_Effects/wall_200_kW_Wall_Plume} \\
\includegraphics[height=2.15in]{SCRIPT_FIGURES/NIST_NRC_Corner_Effects/corner_300_kW_Corner_Plume} &
\includegraphics[height=2.15in]{SCRIPT_FIGURES/NIST_NRC_Corner_Effects/wall_300_kW_Wall_Plume} \\
\includegraphics[height=2.15in]{SCRIPT_FIGURES/NIST_NRC_Corner_Effects/corner_400_kW_Corner_Plume} &
\includegraphics[height=2.15in]{SCRIPT_FIGURES/NIST_NRC_Corner_Effects/wall_400_kW_Wall_Plume}
\end{tabular*}
\caption[NIST/NRC Corner Effects experiments, plume temperature]
{NIST/NRC Corner Effects experiments, plume temperature.}
\label{NIST_NRC_Corner_Plume_Temp}
\end{figure}


\clearpage

\subsection{NIST Pool Fires}
\label{NIST_Pool_Fires_Plume_Temps}

Details of the NIST Pool Fires experiments are found in Section~\ref{NIST_Pool_Fires_Description}.

Fig.~\ref{NIST_Pool_Fires_Temperature} displays the centerline mean temperature profiles for 30.5~cm diameter pool fires of acetone, ethanol, and methanol; and a 37~cm methane burner. Mean and rms centerline and radial temperature profiles for a 100~cm methanol fire are found in Figs.~\ref{NIST_Pool_Fires_Plume_Temps_1} and \ref{NIST_Pool_Fires_Plume_Temps_2}. The radial profiles are located at heights of 20~cm, 60~cm, 100~cm, 140~cm, and 180~cm.

\begin{figure}[!ht]
\begin{tabular*}{\textwidth}{l@{\extracolsep{\fill}}r}
\includegraphics[height=2.15in]{SCRIPT_FIGURES/NIST_Pool_Fires/NIST_Acetone_T_CL} &
\includegraphics[height=2.15in]{SCRIPT_FIGURES/NIST_Pool_Fires/NIST_Ethanol_T_CL} \\
\includegraphics[height=2.15in]{SCRIPT_FIGURES/NIST_Pool_Fires/NIST_Methanol_T_CL} &
\includegraphics[height=2.15in]{SCRIPT_FIGURES/NIST_Pool_Fires/NIST_Methane_T_CL}
\end{tabular*}
\caption[NIST Pool Fires, centerline temperature, acetone, ethanol, methanol, methane, 30 cm]
{NIST Pool Fires, centerline tempertaure profiles, acetone, ethanol, methanol, methane, 30 cm pools.}
\label{NIST_Pool_Fires_Temperature}
\end{figure}


\begin{figure}[!ht]
\begin{tabular*}{\textwidth}{l@{\extracolsep{\fill}}r}
\includegraphics[height=2.15in]{SCRIPT_FIGURES/NIST_Pool_Fires/NIST_Methanol_100_cm_T_CL} &
\includegraphics[height=2.15in]{SCRIPT_FIGURES/NIST_Pool_Fires/NIST_Methanol_100_cm_T_CL_RMS} \\
\includegraphics[height=2.15in]{SCRIPT_FIGURES/NIST_Pool_Fires/NIST_Methanol_100_cm_T_20} &
\includegraphics[height=2.15in]{SCRIPT_FIGURES/NIST_Pool_Fires/NIST_Methanol_100_cm_T_20_RMS} \\
\includegraphics[height=2.15in]{SCRIPT_FIGURES/NIST_Pool_Fires/NIST_Methanol_100_cm_T_60} &
\includegraphics[height=2.15in]{SCRIPT_FIGURES/NIST_Pool_Fires/NIST_Methanol_100_cm_T_60_RMS}
\end{tabular*}
\caption[NIST Pool Fires, 100 cm methanol, plume mean and rms temperatures]
{NIST Pool Fires, 100 cm methanol fire, plume mean and rms temperatures, centerline and radial profiles (20, 60 cm).}
\label{NIST_Pool_Fires_Plume_Temps_1}
\end{figure}

\begin{figure}[p]
\begin{tabular*}{\textwidth}{l@{\extracolsep{\fill}}r}
\includegraphics[height=2.15in]{SCRIPT_FIGURES/NIST_Pool_Fires/NIST_Methanol_100_cm_T_100} &
\includegraphics[height=2.15in]{SCRIPT_FIGURES/NIST_Pool_Fires/NIST_Methanol_100_cm_T_100_RMS} \\
\includegraphics[height=2.15in]{SCRIPT_FIGURES/NIST_Pool_Fires/NIST_Methanol_100_cm_T_140} &
\includegraphics[height=2.15in]{SCRIPT_FIGURES/NIST_Pool_Fires/NIST_Methanol_100_cm_T_140_RMS} \\
\includegraphics[height=2.15in]{SCRIPT_FIGURES/NIST_Pool_Fires/NIST_Methanol_100_cm_T_180} &
\includegraphics[height=2.15in]{SCRIPT_FIGURES/NIST_Pool_Fires/NIST_Methanol_100_cm_T_180_RMS}
\end{tabular*}
\caption[NIST Pool Fires, 100 cm methanol fire, plume mean and rms temperatures]
{NIST Pool Fires, 100 cm methanol fire, plume mean and rms temperatures, radial profiles (100, 140, 180 cm).}
\label{NIST_Pool_Fires_Plume_Temps_2}
\end{figure}


\clearpage

\subsection{NRCC Smoke Tower Experiments, Stairwell Plumes}

The NRCC Smoke Tower experiments include measurements of the temperature of smoke ascending a 10 story stairwell. This data is included here in the chapter on Fire Plumes because smoke movement in a vertical shaft with stairs can be considered an obstructed plume. Shown in Fig.~\ref{NRCC_Smoke_Tower_Stairwell} are predictions of gas temperature measurements made in the center of the stairwell approximately 1.8~m above the slab at floors 2-10. Note that the plot labels ``Slot'' refer to the data acquisition system in the experiments only and have no meaning in the present context. It should be clear from the plot title how the various curves ought to be interpreted.

\newpage

\begin{figure}[p]
\begin{tabular*}{\textwidth}{l@{\extracolsep{\fill}}r}
\includegraphics[height=2.15in]{SCRIPT_FIGURES/NRCC_Smoke_Tower/BK-R_Lower_Stairwell} &
\includegraphics[height=2.15in]{SCRIPT_FIGURES/NRCC_Smoke_Tower/BK-R_Upper_Stairwell} \\
\includegraphics[height=2.15in]{SCRIPT_FIGURES/NRCC_Smoke_Tower/CMP-R_Lower_Stairwell} &
\includegraphics[height=2.15in]{SCRIPT_FIGURES/NRCC_Smoke_Tower/CMP-R_Upper_Stairwell} \\
\includegraphics[height=2.15in]{SCRIPT_FIGURES/NRCC_Smoke_Tower/CLC-I-R_Lower_Stairwell} &
\includegraphics[height=2.15in]{SCRIPT_FIGURES/NRCC_Smoke_Tower/CLC-I-R_Upper_Stairwell} \\
\includegraphics[height=2.15in]{SCRIPT_FIGURES/NRCC_Smoke_Tower/CLC-II-R_Lower_Stairwell} &
\includegraphics[height=2.15in]{SCRIPT_FIGURES/NRCC_Smoke_Tower/CLC-II-R_Upper_Stairwell}
\end{tabular*}
\caption[NRCC Smoke Tower, stairwell temperatures]{NRCC Smoke Tower, stairwell temperatures.}
\label{NRCC_Smoke_Tower_Stairwell}
\end{figure}



\clearpage


\subsection{SP Adiabatic Surface Temperature Experiments}

Three experiments were conducted at SP, Sweden, in 2011, in which a 6~m long, 20~cm diameter vertical column was positioned in the middle of 1.1~m and 1.9~m diesel fuel and 1.1~m heptane
pool fires~\cite{Sjostrom:AST}. Gas, plate, and steel surface temperature measurements were made at heights of 1~m, 2~m, 3~m, 4~m, and 5~m above the pool surface. Gas temperatures were measured with 0.25~mm and 0.50~mm bead thermocouples. The results are very similar and only the 0.25~mm values are used. In the experiments, the fire was reported to lean. The lean was significant for the 1.9~m diesel fuel fire. In that case, only data from 1~m and 2~m above the pool are used. The average temperature between 10~min and 15~min is the basic of comparison.


\begin{figure}[!h]
\begin{tabular*}{\textwidth}{l@{\extracolsep{\fill}}r}
\includegraphics[height=2.15in]{SCRIPT_FIGURES/SP_AST/SP_AST_Diesel_1p1_Gas-1}   &  \includegraphics[height=2.15in]{SCRIPT_FIGURES/SP_AST/SP_AST_Diesel_1p1_Gas-2}    \\
\includegraphics[height=2.15in]{SCRIPT_FIGURES/SP_AST/SP_AST_Diesel_1p1_Gas-3}   &  \includegraphics[height=2.15in]{SCRIPT_FIGURES/SP_AST/SP_AST_Diesel_1p1_Gas-4}     \\
\includegraphics[height=2.15in]{SCRIPT_FIGURES/SP_AST/SP_AST_Diesel_1p1_Gas-5}   &
\end{tabular*}
\caption[SP AST experiments, plume temperature, 1.1 m diesel fire]
{SP AST experiments, plume temperature, 1.1 m diesel fire.}
\label{SP_Diesel_1p1_Gas}
\end{figure}

\newpage

\begin{figure}[p]
\begin{tabular*}{\textwidth}{l@{\extracolsep{\fill}}r}
\includegraphics[height=2.15in]{SCRIPT_FIGURES/SP_AST/SP_AST_Diesel_1p9_Gas-1}   &  \includegraphics[height=2.15in]{SCRIPT_FIGURES/SP_AST/SP_AST_Diesel_1p9_Gas-2}    \\
\includegraphics[height=2.15in]{SCRIPT_FIGURES/SP_AST/SP_AST_Heptane_1p1_Gas-1}  &  \includegraphics[height=2.15in]{SCRIPT_FIGURES/SP_AST/SP_AST_Heptane_1p1_Gas-2}    \\
\includegraphics[height=2.15in]{SCRIPT_FIGURES/SP_AST/SP_AST_Heptane_1p1_Gas-3}  &  \includegraphics[height=2.15in]{SCRIPT_FIGURES/SP_AST/SP_AST_Heptane_1p1_Gas-4}     \\
\includegraphics[height=2.15in]{SCRIPT_FIGURES/SP_AST/SP_AST_Heptane_1p1_Gas-5}  &
\end{tabular*}
\caption[SP AST experiments, plume temperature, 1.9 m diesel and 1.1 m heptane fires]
{SP AST experiments, plume temperature, 1.9 m diesel and 1.9 m heptane fires.}
\label{SP_Diesel_1p9_Gas}
\end{figure}

\clearpage


\subsection{UMD Line Burner}
\label{UMD_Line_Burner_plumes}

In this section, we present thermocouple temperature measurements and computational results for the UMD Line Burner.  Experimental details may be found in White et al.~\cite{White:2015}. FDS simulations are performed at three grid resolutions corresponding to $W/\delta x = 4, 8, 16$, where $W = 5$ cm is the width of the fuel slot in the line burner. Fig.~\ref{UMD_Line_Burner_methane_O2_p18_TC} shows measured and computational results for mean thermocouple temperature across the width of the burner at two heights, $z$, above the burner surface. Fig.~\ref{UMD_Line_Burner_temp_slcf} shows a slice of gas temperature for the case with methane fuel and 18 vol.~\% O$_2$ in the coflow stream (nitrogen dilution).  The purpose of the image is to provide a qualitative result for the flame.

\begin{figure}[h]
\begin{tabular*}{\textwidth}{l@{\extracolsep{\fill}}r}
\includegraphics[height=2.15in]{SCRIPT_FIGURES/UMD_Line_Burner/methane_O2_p18_TC_z_p125} &
\includegraphics[height=2.15in]{SCRIPT_FIGURES/UMD_Line_Burner/methane_O2_p18_TC_z_p250}
\end{tabular*}
\caption[UMD\_Line\_Burner temperature profiles]
{Measured and computed mean thermocouple temperature profiles at 18 vol \% O$_2$.}
\label{UMD_Line_Burner_methane_O2_p18_TC}
\end{figure}

\begin{figure}[h]
\begin{tabular*}{\textwidth}{l@{\extracolsep{\fill}}r}
\hspace{0.25in}\includegraphics[height=4in]{FIGURES/UMD_Line_Burner/methane_dx_p625cm_front} &
\includegraphics[height=4in]{FIGURES/UMD_Line_Burner/methane_dx_p625cm_side}\hspace{0.25in}
\end{tabular*}
\caption[UMD\_Line\_Burner temperature contours]
{UMD Line Burner temperature contours, front (left) and side (right) views for the $\delta x = 0.625$ cm case.  Fuel (natural gas in this case) enters through the red surface.  The air with nitrogen dilution (to 18 vol.~\% O$_2$ in this case) enters through the blue surface.  The white ceramic flame holder is seen surrounding the red burner surface. The right side view corresponds to the profiles shown in Fig.~\ref{UMD_Line_Burner_methane_O2_p18_TC} through the center of the burner at different heights $z$ from the red burner surface.  Within the slice plane blue represents 20 \si{\degree}C, red 1500 \si{\degree}C. }
\label{UMD_Line_Burner_temp_slcf}
\end{figure}

\clearpage

\subsection{USN High Bay Hangar Experiments}

\label{USN_Plume}

A large number of plume temperature measurements are available from the US Navy experiments conducted at Keflavik, Iceland, and Barber's Point, Hawaii. The hangars were very large in size (22~m high in Iceland and 15~m high in Hawaii) and the heat release rates varied from 100~kW to 33~MW. All experiments made use of a fuel pan filled with either JP-5 or JP-8 jet fuel, positioned in the center of the hangar.


\begin{figure}[h!]
\begin{tabular*}{\textwidth}{l@{\extracolsep{\fill}}r}
\includegraphics[height=2.15in]{SCRIPT_FIGURES/USN_Hangars/USN_Iceland_Test_01_1} &
\includegraphics[height=2.15in]{SCRIPT_FIGURES/USN_Hangars/USN_Iceland_Test_02_1} \\
\includegraphics[height=2.15in]{SCRIPT_FIGURES/USN_Hangars/USN_Iceland_Test_03_1} &
\includegraphics[height=2.15in]{SCRIPT_FIGURES/USN_Hangars/USN_Iceland_Test_04_1} \\
\includegraphics[height=2.15in]{SCRIPT_FIGURES/USN_Hangars/USN_Iceland_Test_05_1} &
\includegraphics[height=2.15in]{SCRIPT_FIGURES/USN_Hangars/USN_Iceland_Test_06_1} \\
\end{tabular*}
\caption[USN Hangar experiments, Iceland, plume temperature, Tests 1-6]
{USN Hangar experiments, Iceland, plume temperature, Tests 1-6.}
\label{USN_Plume_Iceland_1}
\end{figure}

\newpage

\begin{figure}[p]
\begin{tabular*}{\textwidth}{l@{\extracolsep{\fill}}r}
\includegraphics[height=2.15in]{SCRIPT_FIGURES/USN_Hangars/USN_Iceland_Test_07_1} &
\includegraphics[height=2.15in]{SCRIPT_FIGURES/USN_Hangars/USN_Iceland_Test_09_1} \\
\includegraphics[height=2.15in]{SCRIPT_FIGURES/USN_Hangars/USN_Iceland_Test_10_1} &
\includegraphics[height=2.15in]{SCRIPT_FIGURES/USN_Hangars/USN_Iceland_Test_11_1} \\
\includegraphics[height=2.15in]{SCRIPT_FIGURES/USN_Hangars/USN_Iceland_Test_12_1} &
\includegraphics[height=2.15in]{SCRIPT_FIGURES/USN_Hangars/USN_Iceland_Test_13_1} \\
\end{tabular*}
\caption[USN Hangar experiments, Iceland, plume temperature, Tests 7, 9-13]
{USN Hangar experiments, Iceland, plume temperature, Tests 7, 9-13.}
\label{USN_Plume_Iceland_2}
\end{figure}

\begin{figure}[p]
\begin{tabular*}{\textwidth}{l@{\extracolsep{\fill}}r}
\includegraphics[height=2.15in]{SCRIPT_FIGURES/USN_Hangars/USN_Iceland_Test_14_1} &
\includegraphics[height=2.15in]{SCRIPT_FIGURES/USN_Hangars/USN_Iceland_Test_15_1} \\
\includegraphics[height=2.15in]{SCRIPT_FIGURES/USN_Hangars/USN_Iceland_Test_17_1} &
\includegraphics[height=2.15in]{SCRIPT_FIGURES/USN_Hangars/USN_Iceland_Test_18_1} \\
\includegraphics[height=2.15in]{SCRIPT_FIGURES/USN_Hangars/USN_Iceland_Test_19_1} &
\includegraphics[height=2.15in]{SCRIPT_FIGURES/USN_Hangars/USN_Iceland_Test_20_1} \\
\end{tabular*}
\caption[USN Hangar experiments, Iceland, plume temperature, Tests 14-15, 17-20]
{USN Hangar experiments, Iceland, plume temperature, Tests 14-15, 17-20.}
\label{USN_Plume_Iceland_3}
\end{figure}

\begin{figure}[p]
\begin{tabular*}{\textwidth}{l@{\extracolsep{\fill}}r}
\includegraphics[height=2.15in]{SCRIPT_FIGURES/USN_Hangars/USN_Hawaii_Test_01_1} &
\includegraphics[height=2.15in]{SCRIPT_FIGURES/USN_Hangars/USN_Hawaii_Test_02_1} \\
\includegraphics[height=2.15in]{SCRIPT_FIGURES/USN_Hangars/USN_Hawaii_Test_03_1} &
\includegraphics[height=2.15in]{SCRIPT_FIGURES/USN_Hangars/USN_Hawaii_Test_04_1} \\
\includegraphics[height=2.15in]{SCRIPT_FIGURES/USN_Hangars/USN_Hawaii_Test_05_1} &
\includegraphics[height=2.15in]{SCRIPT_FIGURES/USN_Hangars/USN_Hawaii_Test_06_1} \\
\includegraphics[height=2.15in]{SCRIPT_FIGURES/USN_Hangars/USN_Hawaii_Test_07_1} &
\includegraphics[height=2.15in]{SCRIPT_FIGURES/USN_Hangars/USN_Hawaii_Test_11_1}
\end{tabular*}
\caption[USN Hangar experiments, Hawaii, plume temperature, Tests 1-7, 11]
{USN Hangar experiments, Hawaii, plume temperature, Tests 1-7, 11.}
\label{USN_Plume_Hawaii}
\end{figure}

\clearpage

\subsection{VTT Large Hall Experiments}

\label{VTT_plume}

The VTT experiments consisted of liquid fuel pan fires positioned in the middle of a large fire test hall. Plume temperatures were measured at two heights above the fire, 6~m (T G.1) and 12~m (T G.2). The flames were observed to extend to about 4~m above the fire pan.


\begin{figure}[!h]
\begin{tabular*}{\textwidth}{l@{\extracolsep{\fill}}r}
\includegraphics[height=2.15in]{SCRIPT_FIGURES/VTT/VTT_01_Plume_Temperature} &
\includegraphics[height=2.15in]{SCRIPT_FIGURES/VTT/VTT_02_Plume_Temperature} \\
\multicolumn{2}{c}{\includegraphics[height=2.15in]{SCRIPT_FIGURES/VTT/VTT_03_Plume_Temperature}}
\end{tabular*}
\caption[VTT experiments, plume temperature]
{VTT experiments, plume temperature.}
\label{VTT_Plume}
\end{figure}


\clearpage

\subsection{Waterloo Methanol Pool Fire Experiment}
\label{Waterloo_Methanol_Plume_Temps}

Figure~\ref{Water_Methanol_Plume_Temp_CL} displays the centerline profile of measured and predicted mean temperatures above a 30~cm diameter methanol pool fire. These measurements were conducted by Hamins and Lock~\cite{Hamins:TN1928} at NIST, along with radial profiles at 40~cm, 50~cm, and 60~cm. All of the other measurements on the following pages were conducted by Weckman at the University of Waterloo~\cite{Weckman:CF1996}.

Figures~\ref{Water_Methanol_Plume_Temp_1} through \ref{Water_Methanol_Plume_Temp_4} display radial profiles of measured and predicted mean (left hand plots) and root mean square (right hand plots) temperatures. The root mean square of the temperature is given by
\be
   \left( \overline{T'T'} \right)^{1/2} = \sqrt{ \frac{ \sum_{i=1}^n (T_i-\overline{T})^2 }{n-1} }
\ee
where $T_i$ is the instantaneous value of temperature and $\overline{T}$ is the average value over 50~s. The profile heights range from 2~cm to 30~cm above the pool surface. Time resolved temperature data was measured using 50~$\mu$m diameter, bare-wire thermocouples (Pt~vs~Pt-10\%~Rh) with known bead diameters in the range of 75~$\mu$m to 100~$\mu$m.

Figures~\ref{Water_Methanol_Tpwp_1} through \ref{Water_Methanol_Tpwp_3} display radial profiles of measured and predicted estimates of temperature-velocity covariances:
\be
   \overline{w'T'} = \frac{ \sum_{i=1}^n (w_i-\overline{w})(T_i-\overline{T}) }{n-1}  \quad ; \quad \overline{u'T'} = \frac{ \sum_{i=1}^n (u_i-\overline{w})(T_i-\overline{T}) }{n-1}
\ee
where $u_i$ and $w_i$ are instantaneous values of the horizontal and vertical components of velocity and $\overline{u}$ and $\overline{w}$ are 50~s time averages.

The FDS results are shown at three grid resolutions, 0.5~cm, 1~cm, and 2~cm.

\begin{figure}[!ht]
\centering
\includegraphics[height=2.15in]{SCRIPT_FIGURES/Waterloo_Methanol/Waterloo_Methanol_Temperature_CL}
\caption[Waterloo Methanol, centerline profile, mean temperature, 2~cm to 60~cm above burner]
{Waterloo Methanol, centerline profile of mean temperature, 2~cm to 60~cm above the burner.}
\label{Water_Methanol_Plume_Temp_CL}
\end{figure}

\begin{figure}[p]
\begin{tabular*}{\textwidth}{l@{\extracolsep{\fill}}r}
\includegraphics[height=2.15in]{SCRIPT_FIGURES/Waterloo_Methanol/Waterloo_Methanol_Temperature_2_cm} &
\includegraphics[height=2.15in]{SCRIPT_FIGURES/Waterloo_Methanol/Waterloo_Methanol_RMS_Temperature_2_cm} \\
\includegraphics[height=2.15in]{SCRIPT_FIGURES/Waterloo_Methanol/Waterloo_Methanol_Temperature_4_cm} &
\includegraphics[height=2.15in]{SCRIPT_FIGURES/Waterloo_Methanol/Waterloo_Methanol_RMS_Temperature_4_cm} \\
\includegraphics[height=2.15in]{SCRIPT_FIGURES/Waterloo_Methanol/Waterloo_Methanol_Temperature_6_cm} &
\includegraphics[height=2.15in]{SCRIPT_FIGURES/Waterloo_Methanol/Waterloo_Methanol_RMS_Temperature_6_cm} \\
\includegraphics[height=2.15in]{SCRIPT_FIGURES/Waterloo_Methanol/Waterloo_Methanol_Temperature_8_cm} &
\includegraphics[height=2.15in]{SCRIPT_FIGURES/Waterloo_Methanol/Waterloo_Methanol_RMS_Temperature_8_cm}
\end{tabular*}
\caption[Waterloo Methanol, radial mean and rms temperature, 2~cm to 8~cm above burner]
{Waterloo Methanol, radial profiles of mean (left) and rms (right) temperature, 2~cm to 8~cm above the burner.}
\label{Water_Methanol_Plume_Temp_1}
\end{figure}

\begin{figure}[p]
\begin{tabular*}{\textwidth}{l@{\extracolsep{\fill}}r}
\includegraphics[height=2.15in]{SCRIPT_FIGURES/Waterloo_Methanol/Waterloo_Methanol_Temperature_10_cm} &
\includegraphics[height=2.15in]{SCRIPT_FIGURES/Waterloo_Methanol/Waterloo_Methanol_RMS_Temperature_10_cm} \\
\includegraphics[height=2.15in]{SCRIPT_FIGURES/Waterloo_Methanol/Waterloo_Methanol_Temperature_12_cm} &
\includegraphics[height=2.15in]{SCRIPT_FIGURES/Waterloo_Methanol/Waterloo_Methanol_RMS_Temperature_12_cm} \\
\includegraphics[height=2.15in]{SCRIPT_FIGURES/Waterloo_Methanol/Waterloo_Methanol_Temperature_14_cm} &
\includegraphics[height=2.15in]{SCRIPT_FIGURES/Waterloo_Methanol/Waterloo_Methanol_RMS_Temperature_14_cm} \\
\includegraphics[height=2.15in]{SCRIPT_FIGURES/Waterloo_Methanol/Waterloo_Methanol_Temperature_16_cm} &
\includegraphics[height=2.15in]{SCRIPT_FIGURES/Waterloo_Methanol/Waterloo_Methanol_RMS_Temperature_16_cm}
\end{tabular*}
\caption[Waterloo Methanol, radial mean and rms temperature, 10~cm to 16~cm above burner]
{Waterloo Methanol, radial profiles of mean (left) and rms (right) temperature, 10~cm to 16~cm above the burner.}
\label{Water_Methanol_Plume_Temp_2}
\end{figure}

\begin{figure}[p]
\begin{tabular*}{\textwidth}{l@{\extracolsep{\fill}}r}
\includegraphics[height=2.15in]{SCRIPT_FIGURES/Waterloo_Methanol/Waterloo_Methanol_Temperature_18_cm} &
\includegraphics[height=2.15in]{SCRIPT_FIGURES/Waterloo_Methanol/Waterloo_Methanol_RMS_Temperature_18_cm} \\
\includegraphics[height=2.15in]{SCRIPT_FIGURES/Waterloo_Methanol/Waterloo_Methanol_Temperature_20_cm} &
\includegraphics[height=2.15in]{SCRIPT_FIGURES/Waterloo_Methanol/Waterloo_Methanol_RMS_Temperature_20_cm} \\
\includegraphics[height=2.15in]{SCRIPT_FIGURES/Waterloo_Methanol/Waterloo_Methanol_Temperature_30_cm} &
\includegraphics[height=2.15in]{SCRIPT_FIGURES/Waterloo_Methanol/Waterloo_Methanol_RMS_Temperature_30_cm} \\
\includegraphics[height=2.15in]{SCRIPT_FIGURES/Waterloo_Methanol/Waterloo_Methanol_Temperature_40_cm} &
\includegraphics[height=2.15in]{SCRIPT_FIGURES/Waterloo_Methanol/Waterloo_Methanol_RMS_Temperature_40_cm}
\end{tabular*}
\caption[Waterloo Methanol, radial mean and rms temperature, 18~cm to 40~cm above burner]
{Waterloo Methanol, radial profiles of mean (left) and rms (right) temperature, 18~cm to 40~cm above the burner. The measurement at 40~cm was performed by Hamins and Lock~\cite{Hamins:TN1928}.}
\label{Water_Methanol_Plume_Temp_3}
\end{figure}

\begin{figure}[p]
\begin{tabular*}{\textwidth}{l@{\extracolsep{\fill}}r}
\includegraphics[height=2.15in]{SCRIPT_FIGURES/Waterloo_Methanol/Waterloo_Methanol_Temperature_50_cm} &
\includegraphics[height=2.15in]{SCRIPT_FIGURES/Waterloo_Methanol/Waterloo_Methanol_RMS_Temperature_50_cm} \\
\includegraphics[height=2.15in]{SCRIPT_FIGURES/Waterloo_Methanol/Waterloo_Methanol_Temperature_60_cm} &
\includegraphics[height=2.15in]{SCRIPT_FIGURES/Waterloo_Methanol/Waterloo_Methanol_RMS_Temperature_60_cm}
\end{tabular*}
\caption[Waterloo Methanol, radial mean and rms temperature, 50~cm to 60~cm above burner]
{Waterloo Methanol, radial profiles of mean (left) and rms (right) temperature, 50~cm to 60~cm above the burner. The measurements were performed by Hamins and Lock~\cite{Hamins:TN1928}.}
\label{Water_Methanol_Plume_Temp_4}
\end{figure}


\begin{figure}[p]
\begin{tabular*}{\textwidth}{l@{\extracolsep{\fill}}r}
\includegraphics[height=2.15in]{SCRIPT_FIGURES/Waterloo_Methanol/Waterloo_Methanol_T_prime_w_prime_2_cm} &
\includegraphics[height=2.15in]{SCRIPT_FIGURES/Waterloo_Methanol/Waterloo_Methanol_T_prime_u_prime_2_cm} \\
\includegraphics[height=2.15in]{SCRIPT_FIGURES/Waterloo_Methanol/Waterloo_Methanol_T_prime_w_prime_4_cm} &
\includegraphics[height=2.15in]{SCRIPT_FIGURES/Waterloo_Methanol/Waterloo_Methanol_T_prime_u_prime_4_cm} \\
\includegraphics[height=2.15in]{SCRIPT_FIGURES/Waterloo_Methanol/Waterloo_Methanol_T_prime_w_prime_6_cm} &
\includegraphics[height=2.15in]{SCRIPT_FIGURES/Waterloo_Methanol/Waterloo_Methanol_T_prime_u_prime_6_cm} \\
\includegraphics[height=2.15in]{SCRIPT_FIGURES/Waterloo_Methanol/Waterloo_Methanol_T_prime_w_prime_8_cm} &
\includegraphics[height=2.15in]{SCRIPT_FIGURES/Waterloo_Methanol/Waterloo_Methanol_T_prime_u_prime_8_cm}
\end{tabular*}
\caption[Waterloo Methanol, radial profiles of $\overline{T'w'}$ and $\overline{T'u'}$, 2~cm to 8~cm above the burner]
{Waterloo Methanol, radial profiles of $\overline{T'w'}$ (left) and $\overline{T'u'}$ (right), 2~cm to 8~cm above the burner.}
\label{Water_Methanol_Tpwp_1}
\end{figure}

\begin{figure}[p]
\begin{tabular*}{\textwidth}{l@{\extracolsep{\fill}}r}
\includegraphics[height=2.15in]{SCRIPT_FIGURES/Waterloo_Methanol/Waterloo_Methanol_T_prime_w_prime_10_cm} &
\includegraphics[height=2.15in]{SCRIPT_FIGURES/Waterloo_Methanol/Waterloo_Methanol_T_prime_u_prime_10_cm} \\
\includegraphics[height=2.15in]{SCRIPT_FIGURES/Waterloo_Methanol/Waterloo_Methanol_T_prime_w_prime_12_cm} &
\includegraphics[height=2.15in]{SCRIPT_FIGURES/Waterloo_Methanol/Waterloo_Methanol_T_prime_u_prime_12_cm} \\
\includegraphics[height=2.15in]{SCRIPT_FIGURES/Waterloo_Methanol/Waterloo_Methanol_T_prime_w_prime_14_cm} &
\includegraphics[height=2.15in]{SCRIPT_FIGURES/Waterloo_Methanol/Waterloo_Methanol_T_prime_u_prime_14_cm} \\
\includegraphics[height=2.15in]{SCRIPT_FIGURES/Waterloo_Methanol/Waterloo_Methanol_T_prime_w_prime_16_cm} &
\includegraphics[height=2.15in]{SCRIPT_FIGURES/Waterloo_Methanol/Waterloo_Methanol_T_prime_u_prime_16_cm}
\end{tabular*}
\caption[Waterloo Methanol, radial profiles of $\overline{T'w'}$ and $\overline{T'u'}$, 10~cm to 16~cm above the burner]
{Waterloo Methanol, radial profiles of $\overline{T'w'}$ (left) and $\overline{T'u'}$ (right), 10~cm to 16~cm above the burner.}
\label{Water_Methanol_Tpwp_2}
\end{figure}

\begin{figure}[p]
\begin{tabular*}{\textwidth}{l@{\extracolsep{\fill}}r}
\includegraphics[height=2.15in]{SCRIPT_FIGURES/Waterloo_Methanol/Waterloo_Methanol_T_prime_w_prime_18_cm} &
\includegraphics[height=2.15in]{SCRIPT_FIGURES/Waterloo_Methanol/Waterloo_Methanol_T_prime_u_prime_18_cm} \\
\includegraphics[height=2.15in]{SCRIPT_FIGURES/Waterloo_Methanol/Waterloo_Methanol_T_prime_w_prime_20_cm} &
\includegraphics[height=2.15in]{SCRIPT_FIGURES/Waterloo_Methanol/Waterloo_Methanol_T_prime_u_prime_20_cm} \\
\includegraphics[height=2.15in]{SCRIPT_FIGURES/Waterloo_Methanol/Waterloo_Methanol_T_prime_w_prime_30_cm} &
\includegraphics[height=2.15in]{SCRIPT_FIGURES/Waterloo_Methanol/Waterloo_Methanol_T_prime_u_prime_30_cm}
\end{tabular*}
\caption[Waterloo Methanol, radial profiles of $\overline{T'w'}$ and $\overline{T'u'}$, 18~cm to 30~cm above the burner]
{Waterloo Methanol, radial profiles of $\overline{T'w'}$ (left) and $\overline{T'u'}$ (right), 18~cm to 30~cm above the burner.}
\label{Water_Methanol_Tpwp_3}
\end{figure}


\clearpage


\subsection{Summary of Plume Temperature Predictions}
\label{Plume Temperature}



\begin{figure}[h!]
\begin{center}
\begin{tabular}{c}
\includegraphics[height=4.0in]{SCRIPT_FIGURES/ScatterPlots/FDS_Plume_Temperature}
\end{tabular}
\end{center}
\caption[Summary of plume temperature predictions]
{Summary of plume temperature predictions.}
\label{Plume_Summary}
\end{figure}

\clearpage


\section{Flame Height}

\subsection{Heskestad's Flame Height Correlation}

Table~\ref{Flame_Height_Parameters} lists the parameters for FDS simulations of fires in a 1~m by 1~m square pan\footnote{The effective diameter, $D$, of a 1~m square pan is 1.13~m, obtained by equating the area of a square and circle.}. Figure~\ref{Flame_Height_check_hrr} shows a verification of the heat release rate for each case, and Fig.~\ref{Flame_Height} compares the FDS predictions with Heskestad's empirical correlation. Note that the flame height for the FDS simulations is defined as the distance above the pan, on average, at which 99~\% of the fuel has been consumed. Note also that the simulations were run at three different grid resolutions. A convenient length scale is given by
\be
   D^* = (Q^*)^{2/5} \, D
\ee
Given a grid cell size, $\dx$, the three resolutions can be characterized by the non-dimensional quantity, $D^*/\dx$, whose values in these cases are 5, 10 and 20.

The flame height definition used in Fig.~\ref{Flame_Height} (99~\% fuel consumption) is admittedly arbitrary and is often questioned when FDS predictions of flame height are compared with experimental values, which are usually based on luminosity (effectively measuring radiation emission from soot).  Further, Heskestad's flame height correlation is one among many such correlations~\cite{SFPE:Heskestad,Steward:1970,Becker:1978,Cox:1985,Hasemi:1984,Cetegen:1984,Delichatsios:1984}, and the reported variation is significant, especially at low values of $Q^*$ where the details of the burner configuration (shape of the burner, etc.) become important.  To illustrate the uncertainty one can expect from FDS calculations and to test the sensitivity of the reported FDS results to the flame height definition, Fig.~\ref{Flame_Height2} shows two different FDS flame height predictions, one at 99~\% fuel consumption (as in Fig.~\ref{Flame_Height})---the red curve---and one using 95~\% fuel consumption---the blue curve.  Three different grid resolutions were run for each flame height definition.  For 99~\% fuel consumption, the red dashed line is the maximum flame height from the three resolutions.  For 95~\% fuel consumption, the blue dashed line is the minimum flame height from the three resolutions.  We also overlay several different flame height correlations (colored solid lines).

Figure~\ref{Flame_Structure} includes comparisons of the predicted HRR as a function of the height of the burner for three different values of $Q^*$. The experimental measurements were performed by Tamanini at Factory Mutual~\cite{Tamanini:CF1983}. Both the HRR and height above the burner have been non-dimensionalized by the total HRR and the flame height, respectively. These results demonstrate that the predicted spatial distribution of the energy release improves as the numerical grid is refined.

\clearpage

\begin{table}[h!]
\caption[Summary of parameters for the flame height predictions]{Summary of parameters for the flame height predictions. The grid cell size, $\dx_{10}$, refers to the case where $D^*/\dx$=10.}
\begin{center}
\begin{tabular}{|c|c|c|c|c|c|}
\hline
$Q^*$       & $\dot{Q}$~(kW) & $D^*$ (m)  & $\dx_5$ (m)  & $\dx_{10}$  & $\dx_{20}$ \\ \hline \hline
0.1         &   151          & 0.45       & 0.090        & 0.045       &  0.022     \\ \hline
0.2         &   303          & 0.59       & 0.119        & 0.059       &  0.030     \\ \hline
0.5         &   756          & 0.86       & 0.171        & 0.086       &  0.043     \\ \hline
1           &   1513         & 1.13       & 0.226        & 0.113       &  0.057     \\ \hline
2           &   3025         & 1.49       & 0.298        & 0.149       &  0.075     \\ \hline
5           &   7564         & 2.15       & 0.430        & 0.215       &  0.108     \\ \hline
10          &   15127        & 2.84       & 0.568        & 0.284       &  0.142     \\ \hline
20          &   30255        & 3.75       & 0.749        & 0.375       &  0.187     \\ \hline
50          &   75636        & 5.40       & 1.081        & 0.540       &  0.270     \\ \hline
100         &   151273       & 7.13       & 1.426        & 0.713       &  0.356     \\ \hline
200         &   302545       & 9.41       & 1.882        & 0.941       &  0.470     \\ \hline
500         &   756363       & 13.6       & 2.715        & 1.357       &  0.679     \\ \hline
1000        &   1512725      & 17.9       & 3.582        & 1.791       &  0.895     \\ \hline
2000        &   3025450      & 23.6       & 4.726        & 2.363       &  1.182     \\ \hline
5000        &   7563625      & 34.1       & 6.819        & 3.409       &  1.705     \\ \hline
10000       &   15127250     & 45.0       & 8.997        & 4.499       &  2.249     \\ \hline
\end{tabular}
\end{center}
\label{Flame_Height_Parameters}
\end{table}

\begin{figure}[!h]
\begin{center}
\includegraphics[height=2.2in]{SCRIPT_FIGURES/Heskestad/Flame_Height_check_hrr}
\end{center}
\caption[Verification of the heat release rate for Heskestad Flame Height cases]
{Verification of the heat release rate for Heskestad Flame Height cases.}
\label{Flame_Height_check_hrr}
\end{figure}

\clearpage

\begin{figure}[!h]
\begin{center}
\includegraphics[height=3in]{SCRIPT_FIGURES/Heskestad/Flame_Height}
\end{center}
\caption[Summary of flame height predictions, Heskestad correlation]
{Comparison of FDS predictions of flame height from a 1~m square pan fire for $Q^*$ values ranging from
0.1 to 10000.}
\label{Flame_Height}
\end{figure}

\begin{figure}[!h]
\begin{center}
\includegraphics[height=3in]{SCRIPT_FIGURES/Heskestad/Flame_Height2}
\end{center}
\caption[Flame height uncertainty, multiple correlations and flame height definitions]
{Flame height predictions from various correlations compared with FDS predictions using two different flame height definitions.  Uncertainty (maximum variation) at $Q^*>1$ is $\pm$15~\%.  At $Q^*=0.1$, the uncertainty is approximately $\pm$65~\%. Correlation references: Steward \cite{Steward:1970}, Becker and Liang \cite{Becker:1978}, Cox and Chitty \cite{Cox:1985}, Heskestad \cite{SFPE:Heskestad}, Hasemi and Tokunaga \cite{Hasemi:1984}, Cetegen \cite{Cetegen:1984}, Delichatsios \cite{Delichatsios:1984}.}
\label{Flame_Height2}
\end{figure}


\begin{figure}[p]
\begin{center}
\begin{tabular}{c}
\includegraphics[height=2.15in]{SCRIPT_FIGURES/Heskestad/Flame_Structure_20} \\
\includegraphics[height=2.15in]{SCRIPT_FIGURES/Heskestad/Flame_Structure_10} \\
\includegraphics[height=2.15in]{SCRIPT_FIGURES/Heskestad/Flame_Structure_5}
\end{tabular}
\end{center}
\caption[Predicted HRR as a function of height above the burner]{Predicted HRR as a function of height above the burner compared to measurements.}
\label{Flame_Structure}
\end{figure}

\clearpage

\subsection{UMD Line Burner}
\label{UMD_Line_Burner_flame_height}

In this section, we present flame height measurements and computational results for the UMD Line Burner.  Experimental details may be found in White et al.~\cite{White:2015}. FDS simulations are performed at three grid resolutions corresponding to $W/\delta x = 4, 8, 16$, where $W = 5$ cm is the width of the fuel slot in the line burner.  Fig.~\ref{UMD_Line_Burner_Lf} shows measured and predicted flame heights of the methane and propane fires as a function of oxygen concentration. For FDS, the flame height is taken as the distance above the burner where 98~\% of the fuel gas is consumed, on average.

\begin{figure}[!h]
\begin{tabular*}{\textwidth}{l@{\extracolsep{\fill}}r}
\includegraphics[height=2.15in]{SCRIPT_FIGURES/UMD_Line_Burner/methane_flame_height} &
\includegraphics[height=2.15in]{SCRIPT_FIGURES/UMD_Line_Burner/propane_flame_height}
\end{tabular*}
\caption[UMD\_Line\_Burner flame height]
{Measured and predicted mean flame heights for the methane and propane UMD Line Burner experiments.}
\label{UMD_Line_Burner_Lf}
\end{figure}




\clearpage

\section{Harrison Spill Plumes/Entrainment Experiments}
\label{Harrison_Spill_Plumes}
\label{Entrainment}

In each of these reduced-scale spill plume experiments, the entrained mass flow rate into the plume was measured at a series of heights by varying the flow through an exhaust hood to maintain a constant smoke layer depth.  Figure~\ref{Entrainment_Plot} compares measured and predicted entrainment rates at five different elevations for the fire scenarios labelled SE4 through SE21 in Ref.~\cite{Harrison:2009}. Two general configurations are considered -- one that is intended to mimic a balcony spill plume and one in which the plume adheres to a vertical wall above the compartment opening.

\begin{figure}[h]
\begin{center}
\includegraphics[height=4in]{SCRIPT_FIGURES/ScatterPlots/FDS_Entrainment}
\caption[Summary of plume entrainment predictions]{A comparison of predicted and measured mass flow rates at various heights for the Harrison Spill Plume experiments.}
\label{Entrainment_Plot}
\end{center}
\end{figure}




\clearpage

\section{Sandia Plume Experiments}

The Fire Laboratory for Accreditation of Models by Experimentation (FLAME) facility \cite{OHern:2005,Blanchat:2001} at Sandia National Laboratories in Albuquerque, New Mexico, is designed specifically for validating models of buoyant fire plumes.  The plume source is 1~m in diameter surrounded by a 0.5~m steel ``ground plane''. Particle Image Velocimetry (PIV) and Planar Laser-Induced Fluorescence (PLIF) techniques were used to obtain instantaneous joint scalar and velocity fields.

\subsection{Sandia 1 m Helium Plume}
\label{Sandia plume}

Calculations of the Sandia 1 m helium plume are run at three grid resolutions: 6 cm, 3 cm, and 1.5 cm.  To give the reader with a qualitative feel for the results, Fig.~\ref{Sandia_He_1m_image} provides a snapshot of density contours from the simulation. The calculations are run in parallel on 16 processors; the outlined blocks indicate the domain decomposition.  Data for vertical velocity, radial velocity, and helium mass fraction are recorded at three levels downstream from the base of the plume, $z = [0.2, 0.4, 0.6]$ m, corresponding to the experimental measurements of O'Hern et al.~\cite{OHern:2005}.  Results for the mean and root mean square (RMS) profiles are given in Figs.~\ref{Sandia_He_1m_velocity} - \ref{Sandia_He_1m_massfraction}.  The means are taken between $t=10$ and $t=20$ seconds in the simulation.

The domain is 3~m by 3~m by 4~m. The boundary conditions are open on all sides with a smooth solid surface surrounding the 1~m diameter helium pool.  The ambient and helium mixture temperature is set to 12~$^\circ$C and the background pressure is set to 80900~Pa to correspond to the experimental conditions.  The helium/acetone/oxygen mixture molecular weight is set to 5.45~kg/kmol.  The turbulent Schmidt and Prandtl numbers are left at the FDS default value of 0.5.  The helium mixture mass flux is specified as 0.0605~kg/s/m$^2$. This case was studied previously by DesJardin et al.~\cite{DesJardin:2004}.
\begin{figure}[h]
\begin{center}
\includegraphics[height=4in]{FIGURES/Sandia_Plumes/Sandia_He_1m_image}
\caption[Sandia 1~m helium plume image]{A snapshot of FDS results at 1.5 cm resolution for the Sandia 1 m helium plume showing density contours.  The rows of measurement devices are visible near the base. The calculations are run in parallel on 16 processors; the outlined blocks indicate the domain decomposition.}
\label{Sandia_He_1m_image}
\end{center}
\end{figure}

\newpage

\begin{figure}[p]
\begin{tabular*}{\textwidth}{l@{\extracolsep{\fill}}r}
\includegraphics[height=2.15in]{SCRIPT_FIGURES/Sandia_Plumes/Sandia_He_1m_W6} &
\includegraphics[height=2.15in]{SCRIPT_FIGURES/Sandia_Plumes/Sandia_He_1m_Wrms_p6} \\
\includegraphics[height=2.15in]{SCRIPT_FIGURES/Sandia_Plumes/Sandia_He_1m_W4} &
\includegraphics[height=2.15in]{SCRIPT_FIGURES/Sandia_Plumes/Sandia_He_1m_Wrms_p4} \\
\includegraphics[height=2.15in]{SCRIPT_FIGURES/Sandia_Plumes/Sandia_He_1m_W2} &
\includegraphics[height=2.15in]{SCRIPT_FIGURES/Sandia_Plumes/Sandia_He_1m_Wrms_p2}
\end{tabular*}
\caption[Sandia 1~m helium plume vertical velocity profiles]
{FDS predictions of mean and root mean square (RMS) vertical velocity profiles for the Sandia 1~m helium plume experiment. Results are shown for 6 cm, 3 cm, and 1.5 cm grid resolutions. With $z$ being the streamwise coordinate, the bottom row is at $z=0.2$ m, the middle row is at $z=0.4$ m, and the top row is at $z=0.6$ m.}
\label{Sandia_He_1m_velocity}
\end{figure}

\begin{figure}[p]
\begin{tabular*}{\textwidth}{l@{\extracolsep{\fill}}r}
\includegraphics[height=2.15in]{SCRIPT_FIGURES/Sandia_Plumes/Sandia_He_1m_U6} &
\includegraphics[height=2.15in]{SCRIPT_FIGURES/Sandia_Plumes/Sandia_He_1m_Urms_p6} \\
\includegraphics[height=2.15in]{SCRIPT_FIGURES/Sandia_Plumes/Sandia_He_1m_U4} &
\includegraphics[height=2.15in]{SCRIPT_FIGURES/Sandia_Plumes/Sandia_He_1m_Urms_p4} \\
\includegraphics[height=2.15in]{SCRIPT_FIGURES/Sandia_Plumes/Sandia_He_1m_U2} &
\includegraphics[height=2.15in]{SCRIPT_FIGURES/Sandia_Plumes/Sandia_He_1m_Urms_p2}
\end{tabular*}
\caption[Sandia 1~m helium plume radial velocity profiles.]
{FDS predictions of mean and root mean square (RMS) radial velocity profiles for the Sandia 1~m helium plume experiment. Results are shown for 6 cm, 3 cm, and 1.5 cm grid resolutions. With $z$ being the streamwise coordinate, the bottom row is at $z=0.2$ m, the middle row is at $z=0.4$ m, and the top row is at $z=0.6$ m.}
\label{Sandia_He_1m_velocity_rms}
\end{figure}

\begin{figure}[p]
\begin{tabular*}{\textwidth}{l@{\extracolsep{\fill}}r}
\includegraphics[height=2.15in]{SCRIPT_FIGURES/Sandia_Plumes/Sandia_He_1m_YHe6} &
\includegraphics[height=2.15in]{SCRIPT_FIGURES/Sandia_Plumes/Sandia_He_1m_Yrms_p6} \\
\includegraphics[height=2.15in]{SCRIPT_FIGURES/Sandia_Plumes/Sandia_He_1m_YHe4} &
\includegraphics[height=2.15in]{SCRIPT_FIGURES/Sandia_Plumes/Sandia_He_1m_Yrms_p4} \\
\includegraphics[height=2.15in]{SCRIPT_FIGURES/Sandia_Plumes/Sandia_He_1m_YHe2} &
\includegraphics[height=2.15in]{SCRIPT_FIGURES/Sandia_Plumes/Sandia_He_1m_Yrms_p2}
\end{tabular*}
\caption[Sandia 1~m helium plume mean and RMS mass fraction profiles.]
{FDS predictions of mean and root mean square (RMS) helium mass fraction profiles for the Sandia 1~m helium plume experiment. Results are shown for 6 cm, 3 cm, and 1.5 cm grid resolutions. With $z$ being the streamwise coordinate, the bottom row shows data at $z=0.2$ m, the middle row shows data at $z=0.4$ m, and the top row shows data at $z=0.6$ m.}
\label{Sandia_He_1m_massfraction}
\end{figure}

\clearpage

\subsection{Sandia 1 m Methane Pool Fire}
\label{Sandia_methane}

The Sandia 1 m methane pool fire series provides data for three methane flow rates: Test 14 (low flow rate), Test~24 (medium flow rate), and Test 17 (high flow rate) \cite{Tieszen:2004}.  The experiments are simulated using three grid resolutions: 6 cm, 3 cm, and 1.5 cm.  Fig.~\ref{Sandia_CH4_1m_image} provides a snapshot of temperature contours from the 1.5 cm Test 17 simulation. The calculations are run in parallel on 16 processors---a similar computational set up as the helium case (the experiments were run in the same facility at Sandia).  Data for vertical velocity and radial velocity are recorded at three levels downstream from the base of the plume, $z = [0.3, 0.5, 0.9]$ m.  Results for the mean profiles (and turbulent kinetic energy for Test 24) are given in Figs.~\ref{Sandia_CH4_1m_Test14_velocity} - \ref{Sandia_CH4_1m_Test17_velocity}.  The means are taken between $t=10$ and $t=20$ seconds in the simulation.

For Test 17, we recorded the vertical velocity as a time series in four locations in the plume---at two positions along the centerline and at two positions on the edge.  The time series from our 1.5 cm simulation at $x=0$ m and $z=0.5$ m, corresponding to Fig.~6 in \cite{Tieszen:2002}, is shown in Fig.~\ref{Sandia_CH4_1m_Test17_spectrum} along with the power spectrum from the average of the four time series locations.  The FDS results compare well with the experimentally obtained puffing frequency of 1.65 Hz \cite{Tieszen:2002}.

\begin{figure}[h]
\begin{center}
\includegraphics[height=4in]{FIGURES/Sandia_Plumes/Sandia_CH4_1m_image}
\caption[Sandia 1~m methane pool fire instantaneous temperature contours.]{A snapshot of FDS results at 1.5 cm resolution for the Sandia 1 m methane pool fire (Test 17 -- high flow rate) showing instantaneous contours of temperature.  The rows of measurement devices (green) are visible near the base.}
\label{Sandia_CH4_1m_image}
\end{center}
\end{figure}

\newpage

\begin{figure}[p]
\begin{tabular*}{\textwidth}{l@{\extracolsep{\fill}}r}
\includegraphics[height=2.15in]{SCRIPT_FIGURES/Sandia_Plumes/Sandia_CH4_1m_Test14_W_zp9} &
\includegraphics[height=2.15in]{SCRIPT_FIGURES/Sandia_Plumes/Sandia_CH4_1m_Test14_U_zp9} \\
\includegraphics[height=2.15in]{SCRIPT_FIGURES/Sandia_Plumes/Sandia_CH4_1m_Test14_W_zp5} &
\includegraphics[height=2.15in]{SCRIPT_FIGURES/Sandia_Plumes/Sandia_CH4_1m_Test14_U_zp5} \\
\includegraphics[height=2.15in]{SCRIPT_FIGURES/Sandia_Plumes/Sandia_CH4_1m_Test14_W_zp3} &
\includegraphics[height=2.15in]{SCRIPT_FIGURES/Sandia_Plumes/Sandia_CH4_1m_Test14_U_zp3}
\end{tabular*}
\caption[Sandia 1~m methane pool fire (Test 14) mean velocity profiles]
{FDS predictions of mean velocity profiles for the Sandia 1~m methane pool fire experiment (Test 14 -- low flow rate). Results are shown for 6 cm, 3 cm, and 1.5 cm grid resolutions. The $z$ coordinate represents height above the methane pool; bottom row: $z=0.3$ m, middle row: $z=0.5$ m, and top row: $z=0.9$ m.}
\label{Sandia_CH4_1m_Test14_velocity}
\end{figure}

\begin{figure}[p]
\begin{tabular*}{\textwidth}{l@{\extracolsep{\fill}}r}
\includegraphics[height=2.15in]{SCRIPT_FIGURES/Sandia_Plumes/Sandia_CH4_1m_Test24_W_zp9} &
\includegraphics[height=2.15in]{SCRIPT_FIGURES/Sandia_Plumes/Sandia_CH4_1m_Test24_U_zp9} \\
\includegraphics[height=2.15in]{SCRIPT_FIGURES/Sandia_Plumes/Sandia_CH4_1m_Test24_W_zp5} &
\includegraphics[height=2.15in]{SCRIPT_FIGURES/Sandia_Plumes/Sandia_CH4_1m_Test24_U_zp5} \\
\includegraphics[height=2.15in]{SCRIPT_FIGURES/Sandia_Plumes/Sandia_CH4_1m_Test24_W_zp3} &
\includegraphics[height=2.15in]{SCRIPT_FIGURES/Sandia_Plumes/Sandia_CH4_1m_Test24_U_zp3}
\end{tabular*}
\caption[Sandia 1~m methane pool fire (Test 24) mean velocity profiles]
{FDS predictions of mean velocity profiles for the Sandia 1~m methane pool fire experiment (Test 24 -- medium flow rate). Results are shown for 6 cm, 3 cm, and 1.5 cm grid resolutions. The $z$ coordinate represents height above the methane pool; bottom row: $z=0.3$ m, middle row: $z=0.5$ m, and top row: $z=0.9$ m.}
\label{Sandia_CH4_1m_Test24_velocity}
\end{figure}

\begin{figure}[p]
\begin{center}
\begin{tabular}{c}
\includegraphics[height=2.15in]{SCRIPT_FIGURES/Sandia_Plumes/Sandia_CH4_1m_Test24_TKE_p9} \\
\includegraphics[height=2.15in]{SCRIPT_FIGURES/Sandia_Plumes/Sandia_CH4_1m_Test24_TKE_p5} \\
\includegraphics[height=2.15in]{SCRIPT_FIGURES/Sandia_Plumes/Sandia_CH4_1m_Test24_TKE_p3}
\end{tabular}
\caption[Sandia 1~m methane pool fire (Test 24) turbulent kinetic energy]
{FDS predictions of turbulent kinetic energy (TKE) profiles for the Sandia 1~m methane pool fire experiment (Test 24 -- medium flow rate). Results are shown for 6 cm, 3 cm, and 1.5 cm grid resolutions. The $z$ coordinate represents height above the methane pool; bottom row: $z=0.3$ m, middle row: $z=0.5$ m, and top row: $z=0.9$ m.}
\label{Sandia_CH4_1m_Test24_tke}
\end{center}
\end{figure}

\begin{figure}[p]
\begin{tabular*}{\textwidth}{l@{\extracolsep{\fill}}r}
\includegraphics[height=2.15in]{SCRIPT_FIGURES/Sandia_Plumes/Sandia_CH4_1m_Test17_W_zp9} &
\includegraphics[height=2.15in]{SCRIPT_FIGURES/Sandia_Plumes/Sandia_CH4_1m_Test17_U_zp9} \\
\includegraphics[height=2.15in]{SCRIPT_FIGURES/Sandia_Plumes/Sandia_CH4_1m_Test17_W_zp5} &
\includegraphics[height=2.15in]{SCRIPT_FIGURES/Sandia_Plumes/Sandia_CH4_1m_Test17_U_zp5} \\
\includegraphics[height=2.15in]{SCRIPT_FIGURES/Sandia_Plumes/Sandia_CH4_1m_Test17_W_zp3} &
\includegraphics[height=2.15in]{SCRIPT_FIGURES/Sandia_Plumes/Sandia_CH4_1m_Test17_U_zp3}
\end{tabular*}
\caption[Sandia 1~m methane pool fire (Test 17) mean velocity profiles]
{FDS predictions of mean velocity profiles for the Sandia 1~m methane pool fire experiment (Test 17). Results are shown for 3 cm and 1.5 cm grid resolutions. The $z$ coordinate represents height above the methane pool; bottom row: $z=0.3$ m, middle row: $z=0.5$ m, and top row: $z=0.9$ m.}
\label{Sandia_CH4_1m_Test17_velocity}
\end{figure}

\begin{figure}[p]
\begin{center}
\begin{tabular}{c}
\includegraphics[height=3.2in]{SCRIPT_FIGURES/Sandia_Plumes/Sandia_CH4_1m_Test17_dx1p5cm_velsignal} \\
\includegraphics[height=3.2in]{SCRIPT_FIGURES/Sandia_Plumes/Sandia_CH4_1m_Test17_dx1p5cm_powerspectrum}
\end{tabular}
\end{center}
\caption[Sandia 1~m methane pool fire velocity signal and power spectrum]
{FDS velocity signal and power spectrum for the Sandia 1~m methane pool fire experiment (Test 17).  The vertical velocity signal (top plot) is output from FDS on the centerline at $z=0.5$ m downstream of the fuel source.  The power spectrum of vertical velocity is measured at four locations and averaged.  Two of the measurement locations are along the centerline, at $z=[0.5, 2.0]$ m, and two are along the edge of the plume, $x = 0.5$ m and $z=[0.5, 2.0]$ m.  The measured puffing frequency of the plume is 1.65 Hz \cite{Tieszen:2002}.  The temporal Nyquist limit of the simulation (the highest resolvable frequency due to the discrete time increment) is $1/(2\delta t) \approx$ 1000 Hz ($\delta t \approx 0.0005$).}
\label{Sandia_CH4_1m_Test17_spectrum}
\end{figure}

\clearpage

\subsection{Sandia 1 m Hydrogen Pool Fire}
\label{Sandia_hydrogen}

Sandia Test 35 \cite{Tieszen:2004} is simulated at three grid resolutions: 6 cm, 3 cm, and 1.5 cm.  The computational set up is nearly identical to the methane cases.  Results for mean vertical and radial velocity are given in Figs.~\ref{Sandia_H2_1m_Test35_velocity}.  Results for turbulent kinetic energy are presented in Fig.~\ref{Sandia_H2_1m_Test35_tke}.  Means are taken from a time average between $t=10$ and $t=20$ seconds in the simulation.

By examining movies of the simulation results we can see a qualitative difference between the methane and hydrogen cases.  The dynamics of the hydrogen case tend to dominated by near total consumption events which create blowback on the pool followed by streaks of accelerating buoyant flow which increase the mean vertical velocity.  An example of the consumption event is seen near the end of the case shown in Fig.~\ref{Sandia_H2_1m_image}.  It is possible that we have not run the simulation long enough for accurate statistics and that streaking events early in the time window (between 10-20 seconds) are biasing the mean vertical velocity to be too high, as is clear from the top-left plot in Fig.~\ref{Sandia_H2_1m_Test35_velocity}.

\begin{figure}[h]
\begin{center}
\includegraphics[height=4in]{FIGURES/Sandia_Plumes/Sandia_H2_1m_image}
\caption[Sandia 1~m hydrogen pool fire instantaneous temperature contours]{A snapshot of FDS results at 1.5 cm resolution for the Sandia 1 m hydrogen pool fire (Test 35) showing instantaneous contours of temperature.  The rows of measurement devices (green) are visible near the base.}
\label{Sandia_H2_1m_image}
\end{center}
\end{figure}

\newpage

\begin{figure}[p]
\begin{tabular*}{\textwidth}{l@{\extracolsep{\fill}}r}
\includegraphics[height=2.15in]{SCRIPT_FIGURES/Sandia_Plumes/Sandia_H2_1m_Test35_W_zp9} &
\includegraphics[height=2.15in]{SCRIPT_FIGURES/Sandia_Plumes/Sandia_H2_1m_Test35_U_zp9} \\
\includegraphics[height=2.15in]{SCRIPT_FIGURES/Sandia_Plumes/Sandia_H2_1m_Test35_W_zp5} &
\includegraphics[height=2.15in]{SCRIPT_FIGURES/Sandia_Plumes/Sandia_H2_1m_Test35_U_zp5} \\
\includegraphics[height=2.15in]{SCRIPT_FIGURES/Sandia_Plumes/Sandia_H2_1m_Test35_W_zp3} &
\includegraphics[height=2.15in]{SCRIPT_FIGURES/Sandia_Plumes/Sandia_H2_1m_Test35_U_zp3}
\end{tabular*}
\caption[Sandia 1~m hydrogen pool fire (Test 35) mean velocity profiles]
{FDS predictions of mean velocity profiles for the Sandia 1~m hydrogen pool fire experiment (Test 35). Results are shown for 6 cm, 3 cm, and 1.5 cm grid resolutions. The $z$ coordinate represents height above the pool; bottom row: $z=0.3$ m, middle row: $z=0.5$ m, and top row: $z=0.9$ m.}
\label{Sandia_H2_1m_Test35_velocity}
\end{figure}

\begin{figure}[p]
\begin{center}
\begin{tabular}{c}
\includegraphics[height=2.15in]{SCRIPT_FIGURES/Sandia_Plumes/Sandia_H2_1m_Test35_TKE_p9} \\
\includegraphics[height=2.15in]{SCRIPT_FIGURES/Sandia_Plumes/Sandia_H2_1m_Test35_TKE_p5} \\
\includegraphics[height=2.15in]{SCRIPT_FIGURES/Sandia_Plumes/Sandia_H2_1m_Test35_TKE_p3}
\end{tabular}
\caption[Sandia 1~m hydrogen pool fire (Test 25) turbulent kinetic energy]
{FDS predictions of turbulent kinetic energy (TKE) profiles for the Sandia 1~m hydrogen pool fire experiment (Test 35). Results are shown for 6 cm, 3 cm, and 1.5 cm grid resolutions. The $z$ coordinate represents height above the methane pool; bottom row: $z=0.3$ m, middle row: $z=0.5$ m, and top row: $z=0.9$ m.}
\label{Sandia_H2_1m_Test35_tke}
\end{center}
\end{figure}


\clearpage

\section{Purdue 7.1 cm Methane Flame}
\label{Purdue_Flames}

Figures \ref{Purdue_7p1_CH4_mixture_fraction}-\ref{Purdue_7p1_CH4_vel_rms} show results for the Purdue 7.1 cm methane flame \cite{Xin:CF2005}.  Three sets of results are presented: two coarse (4 mm) cases and one fine (2 mm) case.  The fine mesh case is run with MPI on 16 meshes.  The coarse cases are run two ways: a single mesh case (dashed lines) and a 16 mesh case (dotted lines).  As should be the case, the single- and multi-mesh cases yield the same results.  This gives confidence in the domain decomposition strategy for open plume flows with FDS.

The discrepancy at the centerline for mixture fraction and vertical velocity may be attributed to our not accounting for (1) the slight divergence of the flow at the burner exit  (7\si{\degree} \cite{Xin:CF2005}) and (2) the asymmetries and fluctuations in the burner exit and ambient environment.  Examination of the FDS output shows the solution remains very symmetric, preventing large gulps of air from penetrating the centerline of the plume, which would tend to smooth out the profiles near the center reducing the bimodal vertical velocity profile and centerline mixture fraction.

\begin{figure}[!h]
\begin{tabular*}{\textwidth}{l@{\extracolsep{\fill}}r}
\includegraphics[height=2.15in]{SCRIPT_FIGURES/Purdue_Flames/7p1_cm_CH4_f_zd_1p41} &
\includegraphics[height=2.15in]{SCRIPT_FIGURES/Purdue_Flames/7p1_cm_CH4_f_zd_p70} \\
\includegraphics[height=2.15in]{SCRIPT_FIGURES/Purdue_Flames/7p1_cm_CH4_f_zd_p14} &
\includegraphics[height=2.15in]{SCRIPT_FIGURES/Purdue_Flames/7p1_cm_CH4_f_zd_p07}
\end{tabular*}
\caption[Purdue 7.1 cm methane flame mean mixture fraction profiles]
{Measured \cite{Zhou:CS1998} and computed radial profiles of the mean mixture fraction at select heights above the burner exit simulated using grids with different spatial resolutions.}
\label{Purdue_7p1_CH4_mixture_fraction}
\end{figure}

\newpage

\begin{figure}[p]
\begin{tabular*}{\textwidth}{l@{\extracolsep{\fill}}r}
\includegraphics[height=2.15in]{SCRIPT_FIGURES/Purdue_Flames/7p1_cm_CH4_T_zd_1p41} &
\includegraphics[height=2.15in]{SCRIPT_FIGURES/Purdue_Flames/7p1_cm_CH4_T_zd_p70} \\
\includegraphics[height=2.15in]{SCRIPT_FIGURES/Purdue_Flames/7p1_cm_CH4_T_zd_p14} &
\includegraphics[height=2.15in]{SCRIPT_FIGURES/Purdue_Flames/7p1_cm_CH4_T_zd_p07}
\end{tabular*}
\caption[Purdue 7.1 cm methane flame mean temperature profiles]
{Inferred \cite{Xin:CF2005} and computed radial profiles of the mean temperature at select heights above the burner exit simulated using grids with different spatial resolutions.}
\label{Purdue_7p1_CH4_temperature}
\end{figure}

\begin{figure}[p]
\begin{tabular*}{\textwidth}{l@{\extracolsep{\fill}}r}
\includegraphics[height=2.15in]{SCRIPT_FIGURES/Purdue_Flames/7p1_cm_CH4_w_zd_p70} &
\includegraphics[height=2.15in]{SCRIPT_FIGURES/Purdue_Flames/7p1_cm_CH4_u_zd_p70} \\
\includegraphics[height=2.15in]{SCRIPT_FIGURES/Purdue_Flames/7p1_cm_CH4_w_zd_p56} &
\includegraphics[height=2.15in]{SCRIPT_FIGURES/Purdue_Flames/7p1_cm_CH4_u_zd_p56} \\
\includegraphics[height=2.15in]{SCRIPT_FIGURES/Purdue_Flames/7p1_cm_CH4_w_zd_p42} &
\includegraphics[height=2.15in]{SCRIPT_FIGURES/Purdue_Flames/7p1_cm_CH4_u_zd_p42}
\end{tabular*}
\caption[Purdue 7.1 cm methane flame mean velocity profiles]
{Measured \cite{Zhou:CS1998} and computed radial profiles of the mean vertical and horizontal velocities at select heights above the burner exit simulated using grids with different spatial resolutions.}
\label{Purdue_7p1_CH4_vertical_velocity}
\end{figure}

\begin{figure}[p]
\begin{tabular*}{\textwidth}{l@{\extracolsep{\fill}}r}
\includegraphics[height=2.15in]{SCRIPT_FIGURES/Purdue_Flames/7p1_cm_CH4_RMS_w_zd_p50} &
\includegraphics[height=2.15in]{SCRIPT_FIGURES/Purdue_Flames/7p1_cm_CH4_RMS_u_zd_p50}
\end{tabular*}
\caption[Purdue 7.1 cm methane flame rms velocity profiles]
{Measured and computed profiles of rms vertical (left) and radial (right) velocity profiles at $z/D = 0.5$.}
\label{Purdue_7p1_CH4_vel_rms}
\end{figure}





\clearpage

\section{FM Vertical Wall Flame Experiments}

Figure~\ref{FM_Vertical_Wall_Flame_temperature} displays the measured and predicted thermocouple (i.e. uncorrected) temperatures as a function of the normal distance, $y$, from the surface of a vertical, water-cooled burner. The measurements were made 771~mm from the base of the burner for propylene mass flow rates of 8.75, 11.85, 12.68, 17.05, 22.37, and 22.49 g/m$^2$/s.

Figure~\ref{FM_Vertical_Wall_Flame_soot_depth} displays the measured and predicted soot depth at heights of 365, 527, 771, 1022, and 1317~mm above the base of the burner for various fuel burning rates. The soot depth was measured by inserting glass rods through the flame, normal to the burner surface, for 2~s. The soot depth is the length of the blackened portion of the rod, taken to represent the average flame depth over the 2~s interval. For the FDS simulations, the soot depth was taken to be the distance from the burner surface where the soot mass fraction drops below 0.0025.

\begin{figure}[p]
\begin{tabular*}{\textwidth}{l@{\extracolsep{\fill}}r}
\includegraphics[height=2.15in]{SCRIPT_FIGURES/FM_Vertical_Wall_Flames/Propylene_Temp_8p75} &
\includegraphics[height=2.15in]{SCRIPT_FIGURES/FM_Vertical_Wall_Flames/Propylene_Temp_11p85} \\
\includegraphics[height=2.15in]{SCRIPT_FIGURES/FM_Vertical_Wall_Flames/Propylene_Temp_12p68} &
\includegraphics[height=2.15in]{SCRIPT_FIGURES/FM_Vertical_Wall_Flames/Propylene_Temp_17p05} \\
\includegraphics[height=2.15in]{SCRIPT_FIGURES/FM_Vertical_Wall_Flames/Propylene_Temp_22p37} &
\includegraphics[height=2.15in]{SCRIPT_FIGURES/FM_Vertical_Wall_Flames/Propylene_Temp_22p49}
\end{tabular*}
\caption[FM Vertical Wall Flame mean temperature profiles]
{Uncorrected horizontal temperature profiles normal to the burner surface for fuel flow rates of 8.75, 11.85, 12.68, 17.05, 22.37, and 22.49 g/m$^2$/s.}
\label{FM_Vertical_Wall_Flame_temperature}
\end{figure}

\begin{figure}[p]
\begin{tabular*}{\textwidth}{l@{\extracolsep{\fill}}r}
\includegraphics[height=2.15in]{SCRIPT_FIGURES/FM_Vertical_Wall_Flames/Soot_Depth_365} &
\includegraphics[height=2.15in]{SCRIPT_FIGURES/FM_Vertical_Wall_Flames/Soot_Depth_527} \\
\includegraphics[height=2.15in]{SCRIPT_FIGURES/FM_Vertical_Wall_Flames/Soot_Depth_771} &
\includegraphics[height=2.15in]{SCRIPT_FIGURES/FM_Vertical_Wall_Flames/Soot_Depth_1022} \\
\multicolumn{2}{c}{\includegraphics[height=2.15in]{SCRIPT_FIGURES/FM_Vertical_Wall_Flames/Soot_Depth_1317}}
\end{tabular*}
\caption[FM Vertical Wall Flame soot depth measurements]
{Soot depth at heights of 365, 527, 771, 1022, and 1317 mm.}
\label{FM_Vertical_Wall_Flame_soot_depth}
\end{figure}







