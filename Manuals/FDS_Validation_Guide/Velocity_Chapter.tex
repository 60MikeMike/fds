% !TEX root = FDS_Validation_Guide.tex

\chapter{Gas Velocity}

Gas velocity is often measured at compartment inlets and outlets as part of a global assessment of mass and
energy conservation.  This chapter contains measurements of gas velocity and related quantities.

\section{ATF Corridor Experiments}

Comparisons of bi-directional velocity measurements with FDS predictions for the ATF Corridor experiments are presented on the following
pages. Velocity measurements were made at four locations, two on the first level (Trees H and I) and two on the second level (Trees J and K).
Shown are the upper-most and lower-most probe for each vertical array. Typically there were four probes per tree, with the number 1 indicating the
upper-most probe.

\newpage

\begin{figure}[p]
\begin{tabular*}{\textwidth}{l@{\extracolsep{\fill}}r}
\includegraphics[height=2.15in]{SCRIPT_FIGURES/ATF_Corridors/ATF_Corridors_Velocity_H_050_kW} &
\includegraphics[height=2.15in]{SCRIPT_FIGURES/ATF_Corridors/ATF_Corridors_Velocity_H_100_kW} \\
 &
\includegraphics[height=2.15in]{SCRIPT_FIGURES/ATF_Corridors/ATF_Corridors_Velocity_H_250_kW} \\
\includegraphics[height=2.15in]{SCRIPT_FIGURES/ATF_Corridors/ATF_Corridors_Velocity_H_500_kW} &
\includegraphics[height=2.15in]{SCRIPT_FIGURES/ATF_Corridors/ATF_Corridors_Velocity_H_Pulsed_HRR}
\end{tabular*}
\caption[ATF Corridors, gas velocity, first level, Location H]{ATF Corridors, gas velocity, first level, Location H.}
\label{ATF_Velocity_H}
\end{figure}

\begin{figure}[p]
\begin{tabular*}{\textwidth}{l@{\extracolsep{\fill}}r}
\includegraphics[height=2.15in]{SCRIPT_FIGURES/ATF_Corridors/ATF_Corridors_Velocity_I_050_kW} &
\includegraphics[height=2.15in]{SCRIPT_FIGURES/ATF_Corridors/ATF_Corridors_Velocity_I_100_kW} \\
\includegraphics[height=2.15in]{SCRIPT_FIGURES/ATF_Corridors/ATF_Corridors_Velocity_I_240_kW} &
\includegraphics[height=2.15in]{SCRIPT_FIGURES/ATF_Corridors/ATF_Corridors_Velocity_I_250_kW} \\
\includegraphics[height=2.15in]{SCRIPT_FIGURES/ATF_Corridors/ATF_Corridors_Velocity_I_500_kW} &
\includegraphics[height=2.15in]{SCRIPT_FIGURES/ATF_Corridors/ATF_Corridors_Velocity_I_Pulsed_HRR}
\end{tabular*}
\caption[ATF Corridors, gas velocity, first level, Location I]{ATF Corridors, gas velocity, first level, Location I.}
\label{ATF_Velocity_I}
\end{figure}

\begin{figure}[p]
\begin{tabular*}{\textwidth}{l@{\extracolsep{\fill}}r}
\includegraphics[height=2.15in]{SCRIPT_FIGURES/ATF_Corridors/ATF_Corridors_Velocity_J_050_kW} &
\includegraphics[height=2.15in]{SCRIPT_FIGURES/ATF_Corridors/ATF_Corridors_Velocity_J_100_kW} \\
\includegraphics[height=2.15in]{SCRIPT_FIGURES/ATF_Corridors/ATF_Corridors_Velocity_J_240_kW} &
\includegraphics[height=2.15in]{SCRIPT_FIGURES/ATF_Corridors/ATF_Corridors_Velocity_J_250_kW} \\
\includegraphics[height=2.15in]{SCRIPT_FIGURES/ATF_Corridors/ATF_Corridors_Velocity_J_500_kW} &
\includegraphics[height=2.15in]{SCRIPT_FIGURES/ATF_Corridors/ATF_Corridors_Velocity_J_Pulsed_HRR}
\end{tabular*}
\caption[ATF Corridors, gas velocity, second level, Location J]{ATF Corridors, gas velocity, second level, Location J.}
\label{ATF_Velocity_J}
\end{figure}

\begin{figure}[p]
\begin{tabular*}{\textwidth}{l@{\extracolsep{\fill}}r}
\includegraphics[height=2.15in]{SCRIPT_FIGURES/ATF_Corridors/ATF_Corridors_Velocity_K_050_kW} &
\includegraphics[height=2.15in]{SCRIPT_FIGURES/ATF_Corridors/ATF_Corridors_Velocity_K_100_kW} \\
\includegraphics[height=2.15in]{SCRIPT_FIGURES/ATF_Corridors/ATF_Corridors_Velocity_K_240_kW} &
\includegraphics[height=2.15in]{SCRIPT_FIGURES/ATF_Corridors/ATF_Corridors_Velocity_K_250_kW} \\
\includegraphics[height=2.15in]{SCRIPT_FIGURES/ATF_Corridors/ATF_Corridors_Velocity_K_500_kW} &
\includegraphics[height=2.15in]{SCRIPT_FIGURES/ATF_Corridors/ATF_Corridors_Velocity_K_Pulsed_HRR}
\end{tabular*}
\caption[ATF Corridors, gas velocity, second level, Location K]{ATF Corridors, gas velocity, second level, Location K.}
\label{ATF_Velocity_K}
\end{figure}


\clearpage

\section{Backward Facing Step}

A snapshot the instantaneous velocity contours of the flow over a backward facing step is shown in Fig.~\ref{fig:vel_slice}. The dimensions of the tunnel are given in Fig.~\ref{tunnel_drawing}. Virtual measurement devices are placed throughout the channel to collect data relating to flow characteristics such as velocity, turbulence RMS velocity, and friction velocity.  These virtual measurement devices are placed into lines, four vertical and one horizontal, with a device in the volumetric center of each grid cell.  A vertical line device is placed within the inlet region at a location of $x=-3h$, and three line devices are placed in the post-step region at locations of $4h$, $6h$, and $10h$. The post-step vertical line devices are placed accordingly to sample the recirculation, reattachment, and recovery regions.  A horizontal line device is used to sample data directly adjacent to the bottom wall of the channel in the post-step region ($0h$ to $20h$).

\begin{figure}[!htb]
    \centering
    \includegraphics[width=\textwidth]{FIGURES/Backward_Facing_Step/backward_facing_step_vel_slice}
    \caption[Instantaneous contours of velocity magnitude]{Instantaneous contours of velocity magnitude.}
    \label{fig:vel_slice}
\end{figure}

The profile of the inlet streamwise velocity component, $\overline{u}(z)$, is specified using experimental data provided by Jovic and Driver~\cite{JD:1994}, while the transverse components, $\overline{v}(z)$ and $\overline{w}(z)$, are set to zero.  Turbulent eddies are injected using the Synthetic Eddy Method of Jarrin~\cite{Jarrin:2008}.  Eddies are injected at random locations in the bottom two inlets, advected with the flow over a distance equal to the maximum eddy length scale and recycled at the inlet.  The inlet maximum eddy length scale is 0.03~m, the number of eddies is 100, and the RMS velocity is set to 0.5~m/s for the middle inlet and 1.0~m/s for the bottom inlet to match the measured inlet data at $x/h=-3$ as closely as possible.

The boundary conditions for velocity and pressure on the top of the domain are ``mirror'', that is, zero gradient.  The spanwise boundaries are periodic.  The outlet boundary is ``open''.

Figure~\ref{fig:friction_pressure_coefficients} shows the longitudinal profiles of the friction coefficient (left) and the pressure coefficient (right).  The $x/h$ location where $C_f$ crosses zero is the reattachment length, a key validation metric for this flow. The measured value is approximately 6. Figure~\ref{fig:flow_profiles} shows the inlet ($x/h=-3$) and downstream mean and covariance profiles.

\begin{figure}[!htb]
    \centering
    \includegraphics[height=2.2in]{SCRIPT_FIGURES/Backward_Facing_Step/backward_facing_step_Cf}
    \includegraphics[height=2.2in]{SCRIPT_FIGURES/Backward_Facing_Step/backward_facing_step_Cp}
    \caption[Friction coefficient and pressure coefficient]{Longitudinal profiles of (left) friction coefficient and (right) pressure coefficient.}
    \label{fig:friction_pressure_coefficients}
\end{figure}

\begin{figure}[!htb]
    \centering
    \includegraphics[height=2.2in]{SCRIPT_FIGURES/Backward_Facing_Step/backward_facing_step_U}
    \includegraphics[height=2.2in]{SCRIPT_FIGURES/Backward_Facing_Step/backward_facing_step_W}\\
    \includegraphics[height=2.2in]{SCRIPT_FIGURES/Backward_Facing_Step/backward_facing_step_uu}
    \includegraphics[height=2.2in]{SCRIPT_FIGURES/Backward_Facing_Step/backward_facing_step_ww}\\
    \includegraphics[height=2.2in]{SCRIPT_FIGURES/Backward_Facing_Step/backward_facing_step_uw}
    \caption[Backward facing step flow profiles]{Flow profiles for various grid resolutions.  Symbols: --$\mathlarger{\mathlarger{\star}}$--, $h/\delta z$=5; --$\Box$--, $h/\delta z$=10; --$\Diamond$--, $h/\delta z$=20; $\mathlarger{\mathlarger{\boldsymbol\circ}}$, J\&D experimental data.}
    \label{fig:flow_profiles}
\end{figure}


\clearpage

\section{Bryant Doorway Experiments}

On the following page there are seven plots comparing the predicted and measured centerline velocity\footnote{Note that the quantity that is being compared is the total velocity multiplied by the sign of its normal component.} profiles in a doorway of a standard ISO~9705 compartment. The measurements shown are based on PIV (Particle Image Velocimetry). Note that some of the measurements do not extend to the top of the doorway (1.96~m above the compartment floor) because the heat from the fire prevented adequate laser resolution of the particles. Velocity measurements were also made using bi-directional probes~\cite{Bryant:FSJ2009}, but these measurements were shown to be up to 20~\% greater in magnitude than the comparable PIV measurement.

\newpage

\begin{figure}[p]
\begin{tabular*}{\textwidth}{l@{\extracolsep{\fill}}r}
\includegraphics[height=2.15in]{SCRIPT_FIGURES/Bryant_Doorway/Bryant_Doorway_034_kW} &
\includegraphics[height=2.15in]{SCRIPT_FIGURES/Bryant_Doorway/Bryant_Doorway_065_kW} \\
\includegraphics[height=2.15in]{SCRIPT_FIGURES/Bryant_Doorway/Bryant_Doorway_096_kW} &
\includegraphics[height=2.15in]{SCRIPT_FIGURES/Bryant_Doorway/Bryant_Doorway_128_kW} \\
\includegraphics[height=2.15in]{SCRIPT_FIGURES/Bryant_Doorway/Bryant_Doorway_160_kW} &
\includegraphics[height=2.15in]{SCRIPT_FIGURES/Bryant_Doorway/Bryant_Doorway_320_kW} \\
\includegraphics[height=2.15in]{SCRIPT_FIGURES/Bryant_Doorway/Bryant_Doorway_511_kW} &
\end{tabular*}
\caption[Bryant Doorway experiments, gas velocity profiles]{Bryant Doorway experiments, gas velocity profiles.}
\label{Bryant_Doorway}
\end{figure}


\clearpage

\section{Edinburgh Vegetation Drag}

The following figures show average velocities, taken over a 60~s period and obtained 225~mm from the leading edge of the fuel bed, as shown in Figure~\ref{Ed_Veg_Drag_Layout}. In the case of the simulations, averaging starts after 20~s in order to allow quasi-steady conditions to establish. The experimental points represent an average of two repeats, involving re-packing of the fuel bed.

\begin{figure}[ht]
\begin{tabular*}{\textwidth}{l@{\extracolsep{\fill}}r}
\includegraphics[height=2.15in]{SCRIPT_FIGURES/Edinburgh_Vegetation_Drag/PineNeedles_20kg_050cms} &
\includegraphics[height=2.15in]{SCRIPT_FIGURES/Edinburgh_Vegetation_Drag/PineNeedles_20kg_100cms} \\
\includegraphics[height=2.15in]{SCRIPT_FIGURES/Edinburgh_Vegetation_Drag/PineNeedles_20kg_150cms} &
\includegraphics[height=2.15in]{SCRIPT_FIGURES/Edinburgh_Vegetation_Drag/PineNeedles_20kg_200cms} \\
\end{tabular*}
\caption[Edinburgh Vegetation Drag, gas velocity profiles, low bulk density]{Edinburgh Vegetation Drag, gas velocity profiles, low bulk density.}
\label{Ed_Veg_Drag_velocity_low}
\end{figure}

\begin{figure}[ht]
\begin{tabular*}{\textwidth}{l@{\extracolsep{\fill}}r}
\includegraphics[height=2.15in]{SCRIPT_FIGURES/Edinburgh_Vegetation_Drag/PineNeedles_40kg_050cms} &
\includegraphics[height=2.15in]{SCRIPT_FIGURES/Edinburgh_Vegetation_Drag/PineNeedles_40kg_100cms} \\
\includegraphics[height=2.15in]{SCRIPT_FIGURES/Edinburgh_Vegetation_Drag/PineNeedles_40kg_150cms} &
\includegraphics[height=2.15in]{SCRIPT_FIGURES/Edinburgh_Vegetation_Drag/PineNeedles_40kg_200cms} \\
\end{tabular*}
\caption[Edinburgh Vegetation Drag, gas velocity profiles, medium bulk density]{Edinburgh Vegetation Drag, gas velocity profiles, medium bulk density.}
\label{Ed_Veg_Drag_velocity_medium}
\end{figure}

\begin{figure}[ht]
\begin{tabular*}{\textwidth}{l@{\extracolsep{\fill}}r}
\includegraphics[height=2.15in]{SCRIPT_FIGURES/Edinburgh_Vegetation_Drag/PineNeedles_60kg_050cms} &
\includegraphics[height=2.15in]{SCRIPT_FIGURES/Edinburgh_Vegetation_Drag/PineNeedles_60kg_100cms} \\
\includegraphics[height=2.15in]{SCRIPT_FIGURES/Edinburgh_Vegetation_Drag/PineNeedles_60kg_150cms} &
\includegraphics[height=2.15in]{SCRIPT_FIGURES/Edinburgh_Vegetation_Drag/PineNeedles_60kg_200cms} \\
\end{tabular*}
\caption[Edinburgh Vegetation Drag, gas velocity profiles, high bulk density]{Edinburgh Vegetation Drag, gas velocity profiles, high bulk density.}
\label{Ed_Veg_Drag_velocity_high}
\end{figure}

\clearpage

\section{FM/FPRF Datacenter Experiments}

On the following page there are eight plots comparing the predicted and measured velocities for the high and low fan speed flow mapping tests in the FM/FPRF datacenter mockup. For each test there are plots for u-velocity, v-velocity, w-velocity and total velocity. Error bars are the measured and predicted RMS values. The dotted lines represent the measurement error. Measurement error was not a simple percentage of the measured value but rather was a propagation of fan flow error (the primary FDS input), sonic anemometer intrinsic error, and an estimate of the error based on the accuracy of placing the anemometer (determined from attempts to make repeat measurements after removing and replacing the probe).

\begin{figure}[ht]
\begin{tabular*}{\textwidth}{l@{\extracolsep{\fill}}r}
\includegraphics[height=3.2in]{SCRIPT_FIGURES/FM_FPRF_Datacenter/FM_Datacenter_Veltest_Low_u} &
\includegraphics[height=3.2in]{SCRIPT_FIGURES/FM_FPRF_Datacenter/FM_Datacenter_Veltest_Low_v} \\
\includegraphics[height=3.2in]{SCRIPT_FIGURES/FM_FPRF_Datacenter/FM_Datacenter_Veltest_Low_w} &
\includegraphics[height=3.2in]{SCRIPT_FIGURES/FM_FPRF_Datacenter/FM_Datacenter_Veltest_Low_vel}
\end{tabular*}
\caption[FM/FPRF experiments, gas velocity, low fan rate]{FM/FPRF experiments, gas velocity, low fan rate.}
\label{FM_Datacenter_Flow_Mapping_1}
\end{figure}

\begin{figure}[ht]
\begin{tabular*}{\textwidth}{l@{\extracolsep{\fill}}r}
\includegraphics[height=3.2in]{SCRIPT_FIGURES/FM_FPRF_Datacenter/FM_Datacenter_Veltest_High_u} &
\includegraphics[height=3.2in]{SCRIPT_FIGURES/FM_FPRF_Datacenter/FM_Datacenter_Veltest_High_v} \\
\includegraphics[height=3.2in]{SCRIPT_FIGURES/FM_FPRF_Datacenter/FM_Datacenter_Veltest_High_w} &
\includegraphics[height=3.2in]{SCRIPT_FIGURES/FM_FPRF_Datacenter/FM_Datacenter_Veltest_High_vel}
\end{tabular*}
\caption[FM/FPRF experiments, gas velocity, high fan rate]{FM/FPRF experiments, gas velocity, high fan rate.}
\label{FM_Datacenter_Flow_Mapping_2}
\end{figure}

\clearpage

\section{McCaffrey's Plume Correlation}

The following plots show the results of simulations of McCaffrey's five fires at three grid resolutions, nominally $D^*/\dx=[3,6,12,24]$ based on the $D^*$ of the 14.4 kW burner (respectively, crude, coarse, medium, and fine resolution).

\begin{figure}[h!]
\begin{tabular*}{\textwidth}{l@{\extracolsep{\fill}}r}
\includegraphics[height=2.15in]{SCRIPT_FIGURES/McCaffrey_Plume/McCaffrey_Plume_Velocity_14_kW} &
\includegraphics[height=2.15in]{SCRIPT_FIGURES/McCaffrey_Plume/McCaffrey_Plume_Velocity_22_kW} \\
\includegraphics[height=2.15in]{SCRIPT_FIGURES/McCaffrey_Plume/McCaffrey_Plume_Velocity_33_kW} &
\includegraphics[height=2.15in]{SCRIPT_FIGURES/McCaffrey_Plume/McCaffrey_Plume_Velocity_45_kW} \\
\multicolumn{2}{c}{\includegraphics[height=2.15in]{SCRIPT_FIGURES/McCaffrey_Plume/McCaffrey_Plume_Velocity_57_kW}}
\end{tabular*}
\caption[McCaffrey Plumes, centerline plume velocity]{McCaffrey Plumes, centerline plume velocity.}
\label{McCaffrey_Plume_Velocity}
\end{figure}

\clearpage

Below we plot the same results but arranged in a different way.  The height dimension is scaled by the fire Froude number and each plot represents nominally the same resolution level.

\begin{figure}[h!]
\begin{tabular*}{\textwidth}{l@{\extracolsep{\fill}}r}
\includegraphics[height=2.15in]{SCRIPT_FIGURES/McCaffrey_Plume/McCaffrey_Velocity_Correlation_Crude} &
\includegraphics[height=2.15in]{SCRIPT_FIGURES/McCaffrey_Plume/McCaffrey_Velocity_Correlation_Coarse} \\
\includegraphics[height=2.15in]{SCRIPT_FIGURES/McCaffrey_Plume/McCaffrey_Velocity_Correlation_Medium} &
\includegraphics[height=2.15in]{SCRIPT_FIGURES/McCaffrey_Plume/McCaffrey_Velocity_Correlation_Fine}
\end{tabular*}
\caption[McCaffrey Plumes, centerline plume velocity, Froude scaling]{McCaffrey Plumes, centerline plume velocity, Froude scaling.}
\label{McCaffrey_Plume_Velocity_Froude}
\end{figure}


\clearpage

\section{NIST Pool Fires}
\label{NIST_Pool_Fires_Velocity}

% THE CL RMS VELOCITIES ARE CURRENTLY UNDER INVESTIGATTION...

% Figures~\ref{NIST_Pool_Fires_Velocity_1} and \ref{NIST_Pool_Fires_Velocity_2} display centerline profiles of mean and rms vertical velocity for 30~cm acetone, ethanol, and methanol pool fires; and 37~cm methane and propane gas burners~\cite{Sung:TN2021}.

% \begin{figure}[!h]
% \begin{tabular*}{\textwidth}{l@{\extracolsep{\fill}}r}
% \includegraphics[height=2.15in]{SCRIPT_FIGURES/NIST_Pool_Fires/NIST_Acetone_w_CL} &
% \includegraphics[height=2.15in]{SCRIPT_FIGURES/NIST_Pool_Fires/NIST_Acetone_w_CL_RMS} \\
% \includegraphics[height=2.15in]{SCRIPT_FIGURES/NIST_Pool_Fires/NIST_Ethanol_w_CL} &
% \includegraphics[height=2.15in]{SCRIPT_FIGURES/NIST_Pool_Fires/NIST_Ethanol_w_CL_RMS} \\
% \includegraphics[height=2.15in]{SCRIPT_FIGURES/NIST_Pool_Fires/NIST_Methane_w_CL} &
% \includegraphics[height=2.15in]{SCRIPT_FIGURES/NIST_Pool_Fires/NIST_Methane_w_CL_RMS}
% \end{tabular*}
% \caption[NIST Pool Fires, centerline velocity]{NIST Pool Fires, centerline profiles of mean and rms vertical velocity for 30~cm acetone and ethanol liquid pool fires; and a 37~cm methane gaseous fire.}
% \label{NIST_Pool_Fires_Velocity_1}
% \end{figure}

% \newpage

% \begin{figure}[p]
% \begin{tabular*}{\textwidth}{l@{\extracolsep{\fill}}r}
% \includegraphics[height=2.15in]{SCRIPT_FIGURES/NIST_Pool_Fires/NIST_Methanol_w_CL} &
% \includegraphics[height=2.15in]{SCRIPT_FIGURES/NIST_Pool_Fires/NIST_Methanol_w_CL_RMS} \\
% \includegraphics[height=2.15in]{SCRIPT_FIGURES/NIST_Pool_Fires/NIST_Propane_20kW_w_CL} &
% \includegraphics[height=2.15in]{SCRIPT_FIGURES/NIST_Pool_Fires/NIST_Propane_20kW_w_CL_RMS} \\
% \includegraphics[height=2.15in]{SCRIPT_FIGURES/NIST_Pool_Fires/NIST_Propane_34kW_w_CL} &
% \includegraphics[height=2.15in]{SCRIPT_FIGURES/NIST_Pool_Fires/NIST_Propane_34kW_w_CL_RMS}
% \end{tabular*}
% \caption[NIST Pool Fires, centerline velocity]{NIST Pool Fires, centerline profiles of mean and rms vertical velocity for a 30~cm methanol fire; and 37~cm propane fires of 20~kW and 34~kW.}
% \label{NIST_Pool_Fires_Velocity_2}
% \end{figure}

Figure~\ref{NIST_Pool_Fires_Velocity_1} displays centerline profiles of mean vertical velocity for 30~cm acetone, ethanol, and methanol pool fires; and 37~cm methane and propane gas burners~\cite{Sung:TN2021}.

\begin{figure}[!h]
\begin{tabular*}{\textwidth}{l@{\extracolsep{\fill}}r}
\includegraphics[height=2.15in]{SCRIPT_FIGURES/NIST_Pool_Fires/NIST_Acetone_w_CL} &
\includegraphics[height=2.15in]{SCRIPT_FIGURES/NIST_Pool_Fires/NIST_Methanol_w_CL} \\
\includegraphics[height=2.15in]{SCRIPT_FIGURES/NIST_Pool_Fires/NIST_Ethanol_w_CL}  &
\includegraphics[height=2.15in]{SCRIPT_FIGURES/NIST_Pool_Fires/NIST_Methane_w_CL} \\
\includegraphics[height=2.15in]{SCRIPT_FIGURES/NIST_Pool_Fires/NIST_Propane_20kW_w_CL} &
\includegraphics[height=2.15in]{SCRIPT_FIGURES/NIST_Pool_Fires/NIST_Propane_34kW_w_CL}
\end{tabular*}
\caption[NIST Pool Fires, centerline velocity]{NIST Pool Fires, centerline profiles of mean vertical velocity for 30~cm acetone, methanol, and ethanol liquid pool fires, a 37~cm methane fire, and propane fires of 20~kW and 34~kW.}
\label{NIST_Pool_Fires_Velocity_1}
\end{figure}

\clearpage

\section{PRISME DOOR Experiments}

Bi-directional probes were placed in the doorway separating the two compartments of the PRISME DOOR experiments. Shown on the plots below are the uppermost and lowest measurement points.

\begin{figure}[!ht]
\begin{tabular*}{\textwidth}{l@{\extracolsep{\fill}}r}
\includegraphics[height=2.15in]{SCRIPT_FIGURES/PRISME/PRS_D1_Velocity} &
\includegraphics[height=2.15in]{SCRIPT_FIGURES/PRISME/PRS_D2_Velocity} \\
\includegraphics[height=2.15in]{SCRIPT_FIGURES/PRISME/PRS_D3_Velocity} &
\includegraphics[height=2.15in]{SCRIPT_FIGURES/PRISME/PRS_D4_Velocity} \\
\includegraphics[height=2.15in]{SCRIPT_FIGURES/PRISME/PRS_D5_Velocity} &
\includegraphics[height=2.15in]{SCRIPT_FIGURES/PRISME/PRS_D6_Velocity}
\end{tabular*}
\caption[PRISME DOOR experiments, gas velocity]{PRISME DOOR experiments, gas velocity.}
\label{PRISME_Velocity}
\end{figure}

\clearpage

\section{Restivo Experiment}

The results of a simulation of Restivo's room ventilation experiment are presented below. To capture the forced inlet flow, the volume near the supply slot needs a fairly fine grid to capture the mixing of air at the shear layer. For the results shown here, the height of the inlet was spanned with 6 grid cells, roughly 3~cm in the vertical dimension, 6~cm in the other two. Finer grids were used in the Musser study~\cite{Musser:1}, but with no appreciable change in results. The component of velocity in the lengthwise direction was measured in four arrays: two vertical arrays located 3~m and 6~m  from the inlet along the centerline of the room, and two horizontal arrays located 8.4~cm above the floor and below the ceiling, respectively. These measurements were taken using hot-wire anemometers. While data on the specific instrumentation used are not readily available, hot-wire systems tend to have limitations at low velocities, with typical thresholds of approximately 0.1~m/s.

\begin{figure}[h!]
\begin{tabular*}{\textwidth}{l@{\extracolsep{\fill}}r}
\includegraphics[height=2.15in]{SCRIPT_FIGURES/Restivo_Experiment/Restivo_3m_Velocity} &
\includegraphics[height=2.15in]{SCRIPT_FIGURES/Restivo_Experiment/Restivo_6m_Velocity} \\
\includegraphics[height=2.15in]{SCRIPT_FIGURES/Restivo_Experiment/Restivo_Ceiling_Velocity} &
\includegraphics[height=2.15in]{SCRIPT_FIGURES/Restivo_Experiment/Restivo_Floor_Velocity}
\end{tabular*}
\caption[Restivo experiment, gas velocity]{Restivo experiment, gas velocity.}
\label{Restivo_Velocity}
\end{figure}

\clearpage

\section{Steckler Compartment Experiments}

Steckler et al.~\cite{Steckler:NBSIR_82-2520} mapped the doorway/window flows in 55 compartment fire experiments. The test matrix is presented in Table~\ref{Steckler_Table}. Shown on the following pages are the centerline velocity profiles, compared with model predictions. Off-center profiles are not considered. The vertical spacing of the measurements was approximately 11~cm, with the uppermost velocity probe centered 5.7~cm below the 10~cm thick soffit. The FDS simulations were uniformly gridded with cells of 5~cm on each side. To quantify the difference between prediction and measurement, the maximum outward velocities, which always occurred at the uppermost measurement location, were compared. It has been found that relatively minor changes in the velocity boundary conditions at the edges and bottom of the door soffit can have a noticeable impact on these results.

\newpage

\begin{figure}[p]
\begin{tabular*}{\textwidth}{l@{\extracolsep{\fill}}r}
\includegraphics[height=2.15in]{SCRIPT_FIGURES/Steckler_Compartment/Steckler_010_Vel} &
\includegraphics[height=2.15in]{SCRIPT_FIGURES/Steckler_Compartment/Steckler_011_Vel} \\
\includegraphics[height=2.15in]{SCRIPT_FIGURES/Steckler_Compartment/Steckler_012_Vel} &
\includegraphics[height=2.15in]{SCRIPT_FIGURES/Steckler_Compartment/Steckler_612_Vel} \\
\includegraphics[height=2.15in]{SCRIPT_FIGURES/Steckler_Compartment/Steckler_013_Vel} &
\includegraphics[height=2.15in]{SCRIPT_FIGURES/Steckler_Compartment/Steckler_014_Vel} \\
\includegraphics[height=2.15in]{SCRIPT_FIGURES/Steckler_Compartment/Steckler_018_Vel} &
\includegraphics[height=2.15in]{SCRIPT_FIGURES/Steckler_Compartment/Steckler_710_Vel}
\end{tabular*}
\caption[Steckler experiments, velocity profiles, Tests 10, 11, 12, 13, 14, 18, 612, 710]{Steckler experiments, velocity profiles, Tests 10, 11, 12, 13, 14, 18, 612, 710.}
\label{Steckler_Vel_1}
\end{figure}

\begin{figure}[p]
\begin{tabular*}{\textwidth}{l@{\extracolsep{\fill}}r}
\includegraphics[height=2.15in]{SCRIPT_FIGURES/Steckler_Compartment/Steckler_810_Vel} &
\includegraphics[height=2.15in]{SCRIPT_FIGURES/Steckler_Compartment/Steckler_016_Vel} \\
\includegraphics[height=2.15in]{SCRIPT_FIGURES/Steckler_Compartment/Steckler_017_Vel} &
\includegraphics[height=2.15in]{SCRIPT_FIGURES/Steckler_Compartment/Steckler_022_Vel} \\
\includegraphics[height=2.15in]{SCRIPT_FIGURES/Steckler_Compartment/Steckler_023_Vel} &
\includegraphics[height=2.15in]{SCRIPT_FIGURES/Steckler_Compartment/Steckler_030_Vel} \\
\includegraphics[height=2.15in]{SCRIPT_FIGURES/Steckler_Compartment/Steckler_041_Vel} &
\includegraphics[height=2.15in]{SCRIPT_FIGURES/Steckler_Compartment/Steckler_019_Vel}
\end{tabular*}
\caption[Steckler experiments, velocity profiles, Tests 16, 17, 19, 22, 23, 30, 41, 810]{Steckler experiments, velocity profiles, Tests 16, 17, 19, 22, 23, 30, 41, 810.}
\label{Steckler_Vel_2}
\end{figure}

\begin{figure}[p]
\begin{tabular*}{\textwidth}{l@{\extracolsep{\fill}}r}
\includegraphics[height=2.15in]{SCRIPT_FIGURES/Steckler_Compartment/Steckler_020_Vel} &
\includegraphics[height=2.15in]{SCRIPT_FIGURES/Steckler_Compartment/Steckler_021_Vel} \\
\includegraphics[height=2.15in]{SCRIPT_FIGURES/Steckler_Compartment/Steckler_114_Vel} &
\includegraphics[height=2.15in]{SCRIPT_FIGURES/Steckler_Compartment/Steckler_144_Vel} \\
\includegraphics[height=2.15in]{SCRIPT_FIGURES/Steckler_Compartment/Steckler_212_Vel} &
\includegraphics[height=2.15in]{SCRIPT_FIGURES/Steckler_Compartment/Steckler_242_Vel} \\
\includegraphics[height=2.15in]{SCRIPT_FIGURES/Steckler_Compartment/Steckler_410_Vel} &
\includegraphics[height=2.15in]{SCRIPT_FIGURES/Steckler_Compartment/Steckler_210_Vel}
\end{tabular*}
\caption[Steckler experiments, velocity profiles, Tests 20, 21, 114, 144, 210, 212, 242, 410]{Steckler experiments, velocity profiles, Tests 20, 21, 114, 144, 210, 212, 242, 410.}
\label{Steckler_Vel_3}
\end{figure}

\begin{figure}[p]
\begin{tabular*}{\textwidth}{l@{\extracolsep{\fill}}r}
\includegraphics[height=2.15in]{SCRIPT_FIGURES/Steckler_Compartment/Steckler_310_Vel} &
\includegraphics[height=2.15in]{SCRIPT_FIGURES/Steckler_Compartment/Steckler_240_Vel} \\
\includegraphics[height=2.15in]{SCRIPT_FIGURES/Steckler_Compartment/Steckler_116_Vel} &
\includegraphics[height=2.15in]{SCRIPT_FIGURES/Steckler_Compartment/Steckler_122_Vel} \\
\includegraphics[height=2.15in]{SCRIPT_FIGURES/Steckler_Compartment/Steckler_224_Vel} &
\includegraphics[height=2.15in]{SCRIPT_FIGURES/Steckler_Compartment/Steckler_324_Vel} \\
\includegraphics[height=2.15in]{SCRIPT_FIGURES/Steckler_Compartment/Steckler_220_Vel} &
\includegraphics[height=2.15in]{SCRIPT_FIGURES/Steckler_Compartment/Steckler_221_Vel}
\end{tabular*}
\caption[Steckler experiments, velocity profiles, Tests 116, 122, 220, 221, 224, 240, 310, 324]{Steckler experiments, velocity profiles, Tests 116, 122, 220, 221, 224, 240, 310, 324.}
\label{Steckler_Vel_4}
\end{figure}

\begin{figure}[p]
\begin{tabular*}{\textwidth}{l@{\extracolsep{\fill}}r}
\includegraphics[height=2.15in]{SCRIPT_FIGURES/Steckler_Compartment/Steckler_514_Vel} &
\includegraphics[height=2.15in]{SCRIPT_FIGURES/Steckler_Compartment/Steckler_544_Vel} \\
\includegraphics[height=2.15in]{SCRIPT_FIGURES/Steckler_Compartment/Steckler_512_Vel} &
\includegraphics[height=2.15in]{SCRIPT_FIGURES/Steckler_Compartment/Steckler_542_Vel} \\
\includegraphics[height=2.15in]{SCRIPT_FIGURES/Steckler_Compartment/Steckler_610_Vel} &
\includegraphics[height=2.15in]{SCRIPT_FIGURES/Steckler_Compartment/Steckler_510_Vel} \\
\includegraphics[height=2.15in]{SCRIPT_FIGURES/Steckler_Compartment/Steckler_540_Vel} &
\includegraphics[height=2.15in]{SCRIPT_FIGURES/Steckler_Compartment/Steckler_517_Vel}
\end{tabular*}
\caption[Steckler experiments, velocity profiles, Tests 510, 512, 514, 517, 540, 542, 544, 610]{Steckler experiments, velocity profiles, Tests 510, 512, 514, 517, 540, 542, 544, 610.}
\label{Steckler_Vel_5}
\end{figure}

\begin{figure}[p]
\begin{tabular*}{\textwidth}{l@{\extracolsep{\fill}}r}
\includegraphics[height=2.15in]{SCRIPT_FIGURES/Steckler_Compartment/Steckler_622_Vel} &
\includegraphics[height=2.15in]{SCRIPT_FIGURES/Steckler_Compartment/Steckler_522_Vel} \\
\includegraphics[height=2.15in]{SCRIPT_FIGURES/Steckler_Compartment/Steckler_524_Vel} &
\includegraphics[height=2.15in]{SCRIPT_FIGURES/Steckler_Compartment/Steckler_541_Vel} \\
\includegraphics[height=2.15in]{SCRIPT_FIGURES/Steckler_Compartment/Steckler_520_Vel} &
\includegraphics[height=2.15in]{SCRIPT_FIGURES/Steckler_Compartment/Steckler_521_Vel} \\
\includegraphics[height=2.15in]{SCRIPT_FIGURES/Steckler_Compartment/Steckler_513_Vel} &
\includegraphics[height=2.15in]{SCRIPT_FIGURES/Steckler_Compartment/Steckler_160_Vel}
\end{tabular*}
\caption[Steckler experiments, velocity profiles, Tests 160, 513, 520, 521, 522, 524, 541, 622]{Steckler experiments, velocity profiles, Tests 160, 513, 520, 521, 522, 524, 541, 622.}
\label{Steckler_Vel_6}
\end{figure}

\begin{figure}[p]
\begin{tabular*}{\textwidth}{l@{\extracolsep{\fill}}r}
\includegraphics[height=2.15in]{SCRIPT_FIGURES/Steckler_Compartment/Steckler_163_Vel} &
\includegraphics[height=2.15in]{SCRIPT_FIGURES/Steckler_Compartment/Steckler_164_Vel} \\
\includegraphics[height=2.15in]{SCRIPT_FIGURES/Steckler_Compartment/Steckler_165_Vel} &
\includegraphics[height=2.15in]{SCRIPT_FIGURES/Steckler_Compartment/Steckler_162_Vel} \\
\includegraphics[height=2.15in]{SCRIPT_FIGURES/Steckler_Compartment/Steckler_167_Vel} &
\includegraphics[height=2.15in]{SCRIPT_FIGURES/Steckler_Compartment/Steckler_161_Vel} \\
\includegraphics[height=2.15in]{SCRIPT_FIGURES/Steckler_Compartment/Steckler_166_Vel} &
\end{tabular*}
\caption[Steckler experiments, velocity profiles, Tests 161, 162, 163, 164, 165, 166, 167]{Steckler experiments, velocity profiles, Tests 161, 162, 163, 164, 165, 166, 167.}
\label{Steckler_Vel_7}
\end{figure}


\clearpage

\section{UL/NIJ House Experiments}

Details of the UL/NIJ Experiments are presented in Section~\ref{UL_NIJ_Description}.

Velocity was measured at (typically) five vertical locations in the open windows and doorways of the two houses. The bi-directional probes were evenly spaced through the height of each opening.

In the single story house, the velocity profiles in the front doorway and Window~E or F are used for comparison. In the two story house, the profiles in the front doorway and Windows~K, L or A are used.

Note that this data has not been included in the summary scatter plot, Fig.~\ref{Steckler_Scatterplot} because it is too noisy to make precise comparisons. It is included here mainly for qualitative comparison.

\begin{figure}[p]
\begin{tabular*}{\textwidth}{l@{\extracolsep{\fill}}r}
\includegraphics[height=2.15in]{SCRIPT_FIGURES/UL_NIJ_Houses/UL_NIJ_Single_Story_Gas_1_BDP_1} &
\includegraphics[height=2.15in]{SCRIPT_FIGURES/UL_NIJ_Houses/UL_NIJ_Single_Story_Gas_1_BDP_6} \\
\includegraphics[height=2.15in]{SCRIPT_FIGURES/UL_NIJ_Houses/UL_NIJ_Single_Story_Gas_2_BDP_1} &
\includegraphics[height=2.15in]{SCRIPT_FIGURES/UL_NIJ_Houses/UL_NIJ_Single_Story_Gas_2_BDP_6} \\
\multicolumn{2}{c}{\includegraphics[height=2.15in]{SCRIPT_FIGURES/UL_NIJ_Houses/UL_NIJ_Single_Story_Gas_5_BDP_5}}
\end{tabular*}
\caption[UL/NIJ Experiments, Velocity, single story house, Tests 1, 2, and 5]{UL/NIJ Experiments, Velocity, single story house, Tests 1, 2, and 5.}
\label{UL_NIJ_Vel_Ranch_123}
\end{figure}

\begin{figure}[p]
\begin{center}
\begin{tabular}{c}
\includegraphics[height=2.15in]{SCRIPT_FIGURES/UL_NIJ_Houses/UL_NIJ_Two_Story_Gas_1_BDP_1} \\
\includegraphics[height=2.15in]{SCRIPT_FIGURES/UL_NIJ_Houses/UL_NIJ_Two_Story_Gas_1_BDP_3} \\
\includegraphics[height=2.15in]{SCRIPT_FIGURES/UL_NIJ_Houses/UL_NIJ_Two_Story_Gas_1_BDP_4}
\end{tabular}
\end{center}
\caption[UL/NIJ Experiments, Velocity, two-story house, Test 1]{UL/NIJ Experiments, Velocity, two-story house, Test 1.}
\label{UL_NIJ_Vel_Colonial_1}
\end{figure}

\begin{figure}[p]
\begin{center}
\begin{tabular}{c}
\includegraphics[height=2.15in]{SCRIPT_FIGURES/UL_NIJ_Houses/UL_NIJ_Two_Story_Gas_4_BDP_1} \\
\includegraphics[height=2.15in]{SCRIPT_FIGURES/UL_NIJ_Houses/UL_NIJ_Two_Story_Gas_4_BDP_3} \\
\includegraphics[height=2.15in]{SCRIPT_FIGURES/UL_NIJ_Houses/UL_NIJ_Two_Story_Gas_4_BDP_4}
\end{tabular}
\end{center}
\caption[UL/NIJ Experiments, Velocity, two-story house, Test 4]{UL/NIJ Experiments, Velocity, two-story house, Test 4.}
\label{UL_NIJ_Vel_Colonial_4}
\end{figure}

\begin{figure}[p]
\begin{center}
\begin{tabular}{c}
\includegraphics[height=2.15in]{SCRIPT_FIGURES/UL_NIJ_Houses/UL_NIJ_Two_Story_Gas_6_BDP_3} \\
\includegraphics[height=2.15in]{SCRIPT_FIGURES/UL_NIJ_Houses/UL_NIJ_Two_Story_Gas_6_BDP_6} \\
\includegraphics[height=2.15in]{SCRIPT_FIGURES/UL_NIJ_Houses/UL_NIJ_Two_Story_Gas_6_BDP_1}
\end{tabular}
\end{center}
\caption[UL/NIJ Experiments, Velocity, two-story house, Test 6]{UL/NIJ Experiments, Velocity, two-story house, Test 6.}
\label{UL_NIJ_Vel_Colonial_6}
\end{figure}


\clearpage


\section{Waterloo Methanol Pool Fire Experiment}
\label{Waterloo_Methanol_Velocity}

Figures~\ref{Water_Methanol_Vert_Vel_1} through \ref{Water_Methanol_Vert_Vel_3} display radial profiles of measured and predicted mean (left hand plots) and root mean square (right hand plots) values of the vertical velocity above a 30~cm diameter methanol pool fire. The root mean square of the vertical velocity is given by:
\be
   \left( \overline{w'w'} \right)^{1/2} = \sqrt{ \frac{ \sum_{i=1}^n (w_i-\overline{w})^2 }{n-1} }
\ee
where $w_i$ is the instantaneous value of the vertical velocity and $\overline{w}$ is the average value over 50~s. The profile heights range from 2~cm to 30~cm above the pool surface. Time resolved velocity measurements were performed using a two component laser doppler anemometer.

Figures~\ref{Water_Methanol_Hori_Vel_1} through \ref{Water_Methanol_Hori_Vel_3} display radial profiles of measured and predicted mean (left hand plots) and root mean square (right hand plots) values of the horizontal velocity.

Figures~\ref{Water_Methanol_upwp_1} through \ref{Water_Methanol_upwp_2} display radial profiles of measured and predicted estimates of the horizontal and vertical velocity covariance:
\be
   \overline{u'w'} = \frac{ \sum_{i=1}^n (u_i-\overline{u})(w_i-\overline{w}) }{n-1}
\ee
where $u_i$ and $w_i$ are instantaneous values of the horizontal and vertical components of velocity and $\overline{u}$ and $\overline{w}$ are 50~s time averages.

The FDS results are shown at three grid resolutions, 0.5~cm, 1~cm, and 2~cm.



\begin{figure}[p]
\begin{tabular*}{\textwidth}{l@{\extracolsep{\fill}}r}
\includegraphics[height=2.15in]{SCRIPT_FIGURES/Waterloo_Methanol/Waterloo_Methanol_Vertical_Velocity_2_cm} &
\includegraphics[height=2.15in]{SCRIPT_FIGURES/Waterloo_Methanol/Waterloo_Methanol_RMS_Vertical_Velocity_2_cm} \\
\includegraphics[height=2.15in]{SCRIPT_FIGURES/Waterloo_Methanol/Waterloo_Methanol_Vertical_Velocity_4_cm} &
\includegraphics[height=2.15in]{SCRIPT_FIGURES/Waterloo_Methanol/Waterloo_Methanol_RMS_Vertical_Velocity_4_cm} \\
\includegraphics[height=2.15in]{SCRIPT_FIGURES/Waterloo_Methanol/Waterloo_Methanol_Vertical_Velocity_6_cm} &
\includegraphics[height=2.15in]{SCRIPT_FIGURES/Waterloo_Methanol/Waterloo_Methanol_RMS_Vertical_Velocity_6_cm} \\
\includegraphics[height=2.15in]{SCRIPT_FIGURES/Waterloo_Methanol/Waterloo_Methanol_Vertical_Velocity_8_cm} &
\includegraphics[height=2.15in]{SCRIPT_FIGURES/Waterloo_Methanol/Waterloo_Methanol_RMS_Vertical_Velocity_8_cm}
\end{tabular*}
\caption[Waterloo Methanol, radial mean and rms vert.~vel., 2~cm to 8~cm above burner]
{Waterloo Methanol, radial profiles of mean and rms vertical velocity, 2~cm to 8~cm above the burner.}
\label{Water_Methanol_Vert_Vel_1}
\end{figure}

\begin{figure}[p]
\begin{tabular*}{\textwidth}{l@{\extracolsep{\fill}}r}
\includegraphics[height=2.15in]{SCRIPT_FIGURES/Waterloo_Methanol/Waterloo_Methanol_Vertical_Velocity_10_cm} &
\includegraphics[height=2.15in]{SCRIPT_FIGURES/Waterloo_Methanol/Waterloo_Methanol_RMS_Vertical_Velocity_10_cm} \\
\includegraphics[height=2.15in]{SCRIPT_FIGURES/Waterloo_Methanol/Waterloo_Methanol_Vertical_Velocity_12_cm} &
\includegraphics[height=2.15in]{SCRIPT_FIGURES/Waterloo_Methanol/Waterloo_Methanol_RMS_Vertical_Velocity_12_cm} \\
\includegraphics[height=2.15in]{SCRIPT_FIGURES/Waterloo_Methanol/Waterloo_Methanol_Vertical_Velocity_14_cm} &
\includegraphics[height=2.15in]{SCRIPT_FIGURES/Waterloo_Methanol/Waterloo_Methanol_RMS_Vertical_Velocity_14_cm} \\
\includegraphics[height=2.15in]{SCRIPT_FIGURES/Waterloo_Methanol/Waterloo_Methanol_Vertical_Velocity_16_cm} &
\includegraphics[height=2.15in]{SCRIPT_FIGURES/Waterloo_Methanol/Waterloo_Methanol_RMS_Vertical_Velocity_16_cm}
\end{tabular*}
\caption[Waterloo Methanol, radial mean and rms vert.~vel., 10~cm to 16~cm above burner]
{Waterloo Methanol, radial profiles of mean and rms vertical velocity, 10~cm to 16~cm above the burner.}
\label{Water_Methanol_Vert_Vel_2}
\end{figure}

\begin{figure}[p]
\begin{tabular*}{\textwidth}{l@{\extracolsep{\fill}}r}
\includegraphics[height=2.15in]{SCRIPT_FIGURES/Waterloo_Methanol/Waterloo_Methanol_Vertical_Velocity_18_cm} &
\includegraphics[height=2.15in]{SCRIPT_FIGURES/Waterloo_Methanol/Waterloo_Methanol_RMS_Vertical_Velocity_18_cm} \\
\includegraphics[height=2.15in]{SCRIPT_FIGURES/Waterloo_Methanol/Waterloo_Methanol_Vertical_Velocity_20_cm} &
\includegraphics[height=2.15in]{SCRIPT_FIGURES/Waterloo_Methanol/Waterloo_Methanol_RMS_Vertical_Velocity_20_cm} \\
\includegraphics[height=2.15in]{SCRIPT_FIGURES/Waterloo_Methanol/Waterloo_Methanol_Vertical_Velocity_30_cm} &
\includegraphics[height=2.15in]{SCRIPT_FIGURES/Waterloo_Methanol/Waterloo_Methanol_RMS_Vertical_Velocity_30_cm}
\end{tabular*}
\caption[Waterloo Methanol, radial mean and rms vert.~vel., 18~cm to 30~cm above burner]
{Waterloo Methanol, radial profiles of mean and rms vertical velocity, 18~cm to 30~cm above the burner.}
\label{Water_Methanol_Vert_Vel_3}
\end{figure}


\begin{figure}[p]
\begin{tabular*}{\textwidth}{l@{\extracolsep{\fill}}r}
\includegraphics[height=2.15in]{SCRIPT_FIGURES/Waterloo_Methanol/Waterloo_Methanol_Horizontal_Velocity_2_cm} &
\includegraphics[height=2.15in]{SCRIPT_FIGURES/Waterloo_Methanol/Waterloo_Methanol_RMS_Horizontal_Velocity_2_cm} \\
\includegraphics[height=2.15in]{SCRIPT_FIGURES/Waterloo_Methanol/Waterloo_Methanol_Horizontal_Velocity_4_cm} &
\includegraphics[height=2.15in]{SCRIPT_FIGURES/Waterloo_Methanol/Waterloo_Methanol_RMS_Horizontal_Velocity_4_cm} \\
\includegraphics[height=2.15in]{SCRIPT_FIGURES/Waterloo_Methanol/Waterloo_Methanol_Horizontal_Velocity_6_cm} &
\includegraphics[height=2.15in]{SCRIPT_FIGURES/Waterloo_Methanol/Waterloo_Methanol_RMS_Horizontal_Velocity_6_cm} \\
\includegraphics[height=2.15in]{SCRIPT_FIGURES/Waterloo_Methanol/Waterloo_Methanol_Horizontal_Velocity_8_cm} &
\includegraphics[height=2.15in]{SCRIPT_FIGURES/Waterloo_Methanol/Waterloo_Methanol_RMS_Horizontal_Velocity_8_cm}
\end{tabular*}
\caption[Waterloo Methanol, radial mean and rms horz.~vel., 2~cm to 8~cm above burner]
{Waterloo Methanol, radial profiles of mean and rms horizontal velocity, 2~cm to 8~cm above the burner.}
\label{Water_Methanol_Hori_Vel_1}
\end{figure}

\begin{figure}[p]
\begin{tabular*}{\textwidth}{l@{\extracolsep{\fill}}r}
\includegraphics[height=2.15in]{SCRIPT_FIGURES/Waterloo_Methanol/Waterloo_Methanol_Horizontal_Velocity_10_cm} &
\includegraphics[height=2.15in]{SCRIPT_FIGURES/Waterloo_Methanol/Waterloo_Methanol_RMS_Horizontal_Velocity_10_cm} \\
\includegraphics[height=2.15in]{SCRIPT_FIGURES/Waterloo_Methanol/Waterloo_Methanol_Horizontal_Velocity_12_cm} &
\includegraphics[height=2.15in]{SCRIPT_FIGURES/Waterloo_Methanol/Waterloo_Methanol_RMS_Horizontal_Velocity_12_cm} \\
\includegraphics[height=2.15in]{SCRIPT_FIGURES/Waterloo_Methanol/Waterloo_Methanol_Horizontal_Velocity_14_cm} &
\includegraphics[height=2.15in]{SCRIPT_FIGURES/Waterloo_Methanol/Waterloo_Methanol_RMS_Horizontal_Velocity_14_cm} \\
\includegraphics[height=2.15in]{SCRIPT_FIGURES/Waterloo_Methanol/Waterloo_Methanol_Horizontal_Velocity_16_cm} &
\includegraphics[height=2.15in]{SCRIPT_FIGURES/Waterloo_Methanol/Waterloo_Methanol_RMS_Horizontal_Velocity_16_cm}
\end{tabular*}
\caption[Waterloo Methanol, radial mean and rms horz.~vel., 10~cm to 16~cm above burner]
{Waterloo Methanol, radial profiles of mean and rms horizontal velocity, 10~cm to 16~cm above the burner.}
\label{Water_Methanol_Hori_Vel_2}
\end{figure}

\begin{figure}[p]
\begin{tabular*}{\textwidth}{l@{\extracolsep{\fill}}r}
\includegraphics[height=2.15in]{SCRIPT_FIGURES/Waterloo_Methanol/Waterloo_Methanol_Horizontal_Velocity_18_cm} &
\includegraphics[height=2.15in]{SCRIPT_FIGURES/Waterloo_Methanol/Waterloo_Methanol_RMS_Horizontal_Velocity_18_cm} \\
\includegraphics[height=2.15in]{SCRIPT_FIGURES/Waterloo_Methanol/Waterloo_Methanol_Horizontal_Velocity_20_cm} &
\includegraphics[height=2.15in]{SCRIPT_FIGURES/Waterloo_Methanol/Waterloo_Methanol_RMS_Horizontal_Velocity_20_cm} \\
\includegraphics[height=2.15in]{SCRIPT_FIGURES/Waterloo_Methanol/Waterloo_Methanol_Horizontal_Velocity_30_cm} &
\includegraphics[height=2.15in]{SCRIPT_FIGURES/Waterloo_Methanol/Waterloo_Methanol_RMS_Horizontal_Velocity_30_cm}
\end{tabular*}
\caption[Waterloo Methanol, radial mean and rms horz.~vel., 18~cm to 30~cm above burner]
{Waterloo Methanol, radial profiles of mean and rms horizontal velocity, 18~cm to 30~cm above the burner.}
\label{Water_Methanol_Hori_Vel_3}
\end{figure}


\begin{figure}[p]
\begin{tabular*}{\textwidth}{l@{\extracolsep{\fill}}r}
\includegraphics[height=2.15in]{SCRIPT_FIGURES/Waterloo_Methanol/Waterloo_Methanol_u_prime_w_prime_2_cm} &
\includegraphics[height=2.15in]{SCRIPT_FIGURES/Waterloo_Methanol/Waterloo_Methanol_u_prime_w_prime_4_cm} \\
\includegraphics[height=2.15in]{SCRIPT_FIGURES/Waterloo_Methanol/Waterloo_Methanol_u_prime_w_prime_6_cm} &
\includegraphics[height=2.15in]{SCRIPT_FIGURES/Waterloo_Methanol/Waterloo_Methanol_u_prime_w_prime_8_cm} \\
\includegraphics[height=2.15in]{SCRIPT_FIGURES/Waterloo_Methanol/Waterloo_Methanol_u_prime_w_prime_10_cm} &
\includegraphics[height=2.15in]{SCRIPT_FIGURES/Waterloo_Methanol/Waterloo_Methanol_u_prime_w_prime_12_cm}
\end{tabular*}
\caption[Waterloo Methanol, radial profiles of $\overline{u'w'}$, 2~cm to 12~cm above the burner]
{Waterloo Methanol, radial profiles of $\overline{u'w'}$, 2~cm to 12~cm above the burner.}
\label{Water_Methanol_upwp_1}
\end{figure}

\begin{figure}[p]
\begin{tabular*}{\textwidth}{l@{\extracolsep{\fill}}r}
\includegraphics[height=2.15in]{SCRIPT_FIGURES/Waterloo_Methanol/Waterloo_Methanol_u_prime_w_prime_14_cm} &
\includegraphics[height=2.15in]{SCRIPT_FIGURES/Waterloo_Methanol/Waterloo_Methanol_u_prime_w_prime_16_cm} \\
\includegraphics[height=2.15in]{SCRIPT_FIGURES/Waterloo_Methanol/Waterloo_Methanol_u_prime_w_prime_18_cm} &
\includegraphics[height=2.15in]{SCRIPT_FIGURES/Waterloo_Methanol/Waterloo_Methanol_u_prime_w_prime_20_cm} \\
\multicolumn{2}{c}{\includegraphics[height=2.15in]{SCRIPT_FIGURES/Waterloo_Methanol/Waterloo_Methanol_u_prime_w_prime_30_cm}}
\end{tabular*}
\caption[Waterloo Methanol, radial profiles of $\overline{u'w'}$, 14~cm to 30~cm above the burner]
{Waterloo Methanol, radial profiles of $\overline{u'w'}$, 14~cm to 30~cm above the burner.}
\label{Water_Methanol_upwp_2}
\end{figure}

\clearpage


\section{WTC Experiments}

Bi-directional probes were positioned inside two of the four inlet openings and three of the four outlet openings. The locations are shown in Fig.~\ref{WTC_velocity_probe_locations}. Exact dimensions are given in Ref.~\cite{NIST_NCSTAR_1-5B}.

\begin{figure}[h!]
\begin{center}
\setlength{\unitlength}{1.6in}
\begin{picture}(3.6,3.82)

\put(0,0){\framebox(3.6,3.82){ }}

\put(0.3,2){\framebox(0.5,0.7){ }}
\put(0.96,2){\framebox(0.75,0.7){ }}
\put(1.87,2){\framebox(0.75,0.7){ }}
\put(2.78,2){\framebox(0.5,0.7){ }}

\put(0.9,1){Below: Inlet vents on the west wall looking inwards.}

\put(0.3,0.1){\framebox(0.5,0.7){ }}
\put(0.96,0.1){\framebox(0.75,0.7){ }}
\put(1.87,0.1){\framebox(0.75,0.7){ }}
\put(2.78,0.1){\framebox(0.5,0.7){ }}

\put(0.9,1.8){Above: Outlet vents on the east wall looking outwards.}

\put(2.13,0.71){\circle*{0.03}}
\put(2.05,0.68){1}
\put(2.13,0.46){\circle*{0.03}}
\put(2.05,0.43){2}
\put(2.13,0.25){\circle*{0.03}}
\put(2.05,0.22){3}

\put(3.07,0.71){\circle*{0.03}}
\put(2.99,0.68){6}
\put(3.07,0.46){\circle*{0.03}}
\put(2.99,0.43){7}
\put(3.07,0.25){\circle*{0.03}}
\put(2.99,0.22){8}

\put(1.47,2.61){\circle*{0.03}}
\put(1.39,2.58){1}
\put(1.47,2.36){\circle*{0.03}}
\put(1.39,2.33){2}
\put(1.47,2.15){\circle*{0.03}}
\put(1.39,2.12){3}

\put(1.20,2.61){\circle*{0.03}}
\put(1.12,2.58){4}
\put(1.20,2.15){\circle*{0.03}}
\put(1.12,2.12){5}

\put(0.54,2.61){\circle*{0.03}}
\put(0.46,2.58){6}
\put(0.54,2.36){\circle*{0.03}}
\put(0.46,2.33){7}
\put(0.54,2.15){\circle*{0.03}}
\put(0.46,2.12){8}

\put(3.06,2.61){\circle*{0.03}}
\put(2.98,2.58){9}
\put(3.06,2.15){\circle*{0.03}}
\put(2.94,2.12){10}

\end{picture}
\end{center}
\caption[Layout of velocity probes, WTC Experiments]{Layout of the bi-directional probes in the inlet (west wall) and outlet (east wall) vents, WTC Experiments.}
\label{WTC_velocity_probe_locations}
\end{figure}

\newpage

\begin{figure}[p]
\begin{tabular*}{\textwidth}{l@{\extracolsep{\fill}}r}
\includegraphics[height=2.15in]{SCRIPT_FIGURES/WTC/WTC_01_velo_1-3} &
\includegraphics[height=2.15in]{SCRIPT_FIGURES/WTC/WTC_02_velo_1-3} \\
\includegraphics[height=2.15in]{SCRIPT_FIGURES/WTC/WTC_03_velo_1-3} &
\includegraphics[height=2.15in]{SCRIPT_FIGURES/WTC/WTC_04_velo_1-3} \\
\includegraphics[height=2.15in]{SCRIPT_FIGURES/WTC/WTC_05_velo_1-3} &
\includegraphics[height=2.15in]{SCRIPT_FIGURES/WTC/WTC_06_velo_1-3}
\end{tabular*}
\caption[WTC experiments, inlet velocity, Points 1-3]{WTC experiments, inlet velocity, Points 1-3.}
\label{WTC_velo_1-3}
\end{figure}

\begin{figure}[p]
\begin{tabular*}{\textwidth}{l@{\extracolsep{\fill}}r}
\includegraphics[height=2.15in]{SCRIPT_FIGURES/WTC/WTC_01_velo_6-8} &
\includegraphics[height=2.15in]{SCRIPT_FIGURES/WTC/WTC_02_velo_6-8} \\
\includegraphics[height=2.15in]{SCRIPT_FIGURES/WTC/WTC_03_velo_6-8} &
\includegraphics[height=2.15in]{SCRIPT_FIGURES/WTC/WTC_04_velo_6-8} \\
\includegraphics[height=2.15in]{SCRIPT_FIGURES/WTC/WTC_05_velo_6-8} &
\includegraphics[height=2.15in]{SCRIPT_FIGURES/WTC/WTC_06_velo_6-8}
\end{tabular*}
\caption[WTC experiments, inlet velocity, Points 6-8]{WTC experiments, inlet velocity, Points 6-8.}
\label{WTC_velo_6-8}
\end{figure}

\begin{figure}[p]
\begin{tabular*}{\textwidth}{l@{\extracolsep{\fill}}r}
\includegraphics[height=2.15in]{SCRIPT_FIGURES/WTC/WTC_01_velo_out_1-5} &
\includegraphics[height=2.15in]{SCRIPT_FIGURES/WTC/WTC_02_velo_out_1-5} \\
\includegraphics[height=2.15in]{SCRIPT_FIGURES/WTC/WTC_03_velo_out_1-5} &
\includegraphics[height=2.15in]{SCRIPT_FIGURES/WTC/WTC_04_velo_out_1-5} \\
\includegraphics[height=2.15in]{SCRIPT_FIGURES/WTC/WTC_05_velo_out_1-5} &
\includegraphics[height=2.15in]{SCRIPT_FIGURES/WTC/WTC_06_velo_out_1-5}
\end{tabular*}
\caption[WTC experiments, outlet velocity, Points 1-5]{WTC experiments, outlet velocity, Points 1-5.}
\label{WTC_velo_out_1-5}
\end{figure}

\begin{figure}[p]
\begin{tabular*}{\textwidth}{l@{\extracolsep{\fill}}r}
\includegraphics[height=2.15in]{SCRIPT_FIGURES/WTC/WTC_01_velo_out_6-10} &
\includegraphics[height=2.15in]{SCRIPT_FIGURES/WTC/WTC_02_velo_out_6-10} \\
\includegraphics[height=2.15in]{SCRIPT_FIGURES/WTC/WTC_03_velo_out_6-10} &
\includegraphics[height=2.15in]{SCRIPT_FIGURES/WTC/WTC_04_velo_out_6-10} \\
\includegraphics[height=2.15in]{SCRIPT_FIGURES/WTC/WTC_05_velo_out_6-10} &
\includegraphics[height=2.15in]{SCRIPT_FIGURES/WTC/WTC_06_velo_out_6-10}
\end{tabular*}
\caption[WTC experiments, outlet velocity, Points 6-10]{WTC experiments, outlet velocity, Points 6-10.}
\label{WTC_velo_out_6-10}
\end{figure}



\clearpage

\section{Summary of Velocity Predictions}
\label{Velocity}

\begin{figure}[h!]
\begin{center}
\begin{tabular}{l}
\includegraphics[height=4.0in]{SCRIPT_FIGURES/ScatterPlots/FDS_Velocity}
\end{tabular}
\end{center}
\caption[Summary of velocity predictions]
{Summary of comparisons of predicted and measured maximum velocities.}
\label{Steckler_Scatterplot}
\end{figure}

